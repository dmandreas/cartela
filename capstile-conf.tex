%---COLOR---%
\definecolor{rubrica}{RGB}{200,12,12} %color rojo de las rubricas
\definecolor{letra}{RGB}{3,142,3} %color verde de las lettrine
\definecolor{neg}{RGB}{0,0,0} %color negro absoluto
\definecolor{azul}{RGB}{0,0,255}

%\renewcommand{\chaptermark}[1]{ \markboth{#1}{}}%este cogigo es para quitar la palabra capítulo de la cabecera cuado se está escribiendo en la clase book.

%---separacion columnas multicol----%
\setlength{\columnsep}{1cm}

%---------------------------COMANDOS LECTURAS------------------%
%titulo para fiesta
\newcommand{\fest}[2]{%
\subsection[#1]{#1 \\* #2
%\vspace{4 mm}%
%\begin{center}%
%{\large\scshape{\color{rubrica}#1}\\*{\color{azul}#2}}%
%\end{center}%
%\vspace{-6 mm}%
}}
%titulo para no fiesta
\newcommand{\festnoday}[1]{%
\vspace{4 mm}%
\pagebreak[3]%
\begin{center}%
{\large\scshape{\color{azul}#1}}
\end{center}%
\vspace{-6 mm}%
}

%para empezar la lectura:
\newcommand{\lectu}[2]{%
\subsubsection{#1{\hfill{\color{rubrica}#2}}}%
}

\newcommand{\lect}[2]{%[3]
\subsubsection{#1{\hfill{\color{rubrica}#2}}}%
%\begin{samepage}%
%\vspace{8 mm}%
%\noindent{#1{\hfill{\color{rubrica}#2}}}%
%\vspace{3 mm}%
%\\*%
%\noindent{#3}%
%\end{samepage}%
}
%-----------------VERSICULUM-----------------------%
\newcommand{\va}{{\color{rubrica}\Vbar . }}
\newcommand{\ra}{{\color{rubrica}\Rbar . }}
\newcommand{\rub}[1]{{\fontsize{12}{14}\selectfont {\color{rubrica}\textit{#1}}  \par}}%comando para poner letra roja, en tamaño 12, con espacio 14 y italica.


%----lettrine---%
\newcommand{\ldos}[1]{\lettrine[lines=2]{\color{letra}\zall{#1}}{}}
\newcommand{\ltres}[1]{\lettrine[lines=3]{\color{letra}\zall{#1}}{}}

%%---- cambiar margen ----- %%
\newenvironment{changemargin}[2]{%
\begin{list}{}{%
\setlength{\topsep}{0pt}%
\setlength{\leftmargin}{#1}%
\setlength{\rightmargin}{#2}%
\setlength{\listparindent}{\parindent}%
\setlength{\itemindent}{\parindent}%
\setlength{\parsep}{\parskip}%
}%
\item[]}{\end{list}}
%se escribe: begin{changemargin}{xcm}{xcm} y se acaba end{changemargin}
%-----------------------------------------------%
%%------PARA ESCRIBIR EN PARALELO BILINGÜE----%%
\usepackage{parcolumns}
\usepackage{etoolbox}
\makeatletter%para dibujar la linea roja. Si no, se pone negra.
\patchcmd{\pc@placeboxes}{\vrule}{{\VRULE}}{}{}
\makeatother
\newcommand\VRULE{\color{red}\vrule width 0.5pt}

%comando para escribir solo ``\PC{...}{...}'', en vez de todo el codigo:
\newcommand\PC[2]{%
\pretolerance=150%
\tolerance=753%
\hbadness=752%
\hfuzz0pt%
\emergencystretch=0em%
\begin{parcolumns}[nofirstindent, rulebetween]{2}%
\colchunk[1]{\selectlanguage{latin}#1}%
\colchunk[2]{\selectlanguage{swedish}#2}%
\colplacechunks%
\end{parcolumns}% 
}%\addvspace{0.3\baselineskip}}
\newcommand\bistart{\begin{parcolumns}[nofirstindent, rulebetween]{2}}%
\newcommand\lat[1]{\colchunk[1]{#1}}%
\newcommand\swe[1]{\colchunck[2]{#1}%
					\par
					\colplacechunks}%
\newcommand\biend{	\end{parcolumns}}%
%-----------------------------------------------%

%para escribir parentesis en rojo:
\newcommand{\parent}[1]{%
	{\color{rubrica}(}%
	{#1}%
	{\color{rubrica})}%
	}%
	
	
%Para las letanias de los santos:
\newcommand{\letai}[4]{% Para un santo
 Sancte {#1}%
\textbf{#2}%
\textit{#3}%
{#4}%
,&\ra ora pro nobis.%
\\}
\newcommand{\letaii}[3]{%
{#1}%
\textbf{#2}%
\textit{#3}%
,&\ra orate pro nobis.%
\\}
\newcommand{\leta}[4]{%
Sancta {#1}%
\textbf{#2}%
\textit{#3}%
{#4}%
,&\ra ora pro nobis.%
\\}
\parindent 0mm