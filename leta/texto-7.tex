%leta

% !TEX program = LuaLaTeX+se
% !TEX encoding = UTF-8 Unicode
%\documentclass{report}
\documentclass[twoside,openany,a4paper]{memoir}

\usepackage[a4paper,margin=2cm]{geometry}
\usepackage{fontspec}
\usepackage[autocompile]{gregoriotex}
\usepackage[savepos]{zref}
\usepackage{lettrine}
\usepackage{amssymb}%activa el modo matematico. Lo hago para poder escribir la cruz maltese de bendecir que se escribe "$\maltese$"
\usepackage{Zallman}%aqui elijo el tipo de letra para lettrine
\usepackage{libertine}
%\usepackage{multicol}

%\pagestyle{empty}
%\columnseprule=0.4pt

\input{configuracionletania.tex}
%\setmainfont{kurier-Regular.otf}
\begin{document}
hola mundo

\Vbar

\Rbar

\begin{minipage}{10,5cm}%
\gregorioscore{Litany-I}%
\end{minipage}%

%\begin{nstabbing}
%\>\Vbar \>Fili \>Re-\>démp-\>tor \>mundi \>\textbf{De}-\>us, \>mi-\>se-\>\textit{ré}-\>\textit{re} \>no-\>bis.\\
%\>{\large \Vbar .} \>Ut \>dígni \>efficiá-\>mur \>pro-\>mi-\>sio-\>ni-\>bus \>Chrís-\>\>ti.\\
%\>Sancta \>\>Trínitas \>\>unus \>\textbf{De}-\>us, \>mi-\>se-\>\textit{ré}-\>\textit{re} \>no-\>bis.\\
%\end{nstabbing}

\begin{minipage}{12cm}%
\gregorioscore{Litany-II}%
\end{minipage}%

%\begin{nstabbing}
%\>\Vbar. \>Ut \>dígni \>efficiá-\>mur \>pro-\>mi-\>sio-\>ni-\>bus \>Chrís-\>\>ti.\\
%\end{nstabbing}
\begin{minipage}[t]{8cm}%
	\gabcsnippet{(c4) V/. Ky(f)ri(d)e(d) e(d)le(d)i(c)son(d.) (::)}%
\end{minipage}%
\end{document}