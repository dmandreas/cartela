%Estilo fancy%%%

%---FANCYHDR---%

%---------DEFINE EL ESTILO DE LOS ENCABEZADOS DE LAS PAGINAS--------------%
\pagestyle{fancy} 
\setlength\headheight{18.45483pt}
\fancyhf{} 
\fancyhead[LO]{\small\color{rubrica}\rightmark} % En las páginas impares, parte izquierda del encabezado, aparecerá el nombre de capítulo 
\fancyhead[RE]{\small\color{rubrica}\leftmark} % En las páginas pares, parte derecha del encabezado, aparecerá el nombre de sección 
\fancyhead[RO,LE]{\thepage} % Números de página en las esquinas de los encabezados 
%\fancyfoot[CE,CO]{} %Escribo este texto a la izquierda en laspáginas impares y a la derecha en las pares 
%\renewcommand{\footrulewidth}{0.3pt}


%-------------DEFINE EL ESTILO DEL CAPITULO-------------------%
\makeatletter
\def\thickhrulefill{\leavevmode \leaders \hrule height 1ex \hfill \kern \z@}
\def\@makechapterhead#1{%
  %\vspace*{10\p@}%
  {\parindent \z@ \centering \reset@font
        {\Large\bfseries \thechapter }
        \par\nobreak
        \vspace*{10\p@}%
        \interlinepenalty\@M
        \hrule
        \vspace*{10\p@}%
        \Large \bfseries #1\par\nobreak
        \par
        \vspace*{10\p@}%
    \vskip 10\p@
  }}

%---------Para definir el estilo de la seccion, y subseccion--------%
\setsecheadstyle{\LARGE\centering\scshape\color{rubrica}}
\setsubsecheadstyle{\color{rubrica}\Large\centering\scshape}
%\setsubsubsecheadstyle{\large\centering\scshape}


