\chapter{DEN SAKRAMENTALA VÄLSIGNELSENS STRUKTUR}

\markboth{SAKRAMENTAL VÄLSIGNESE}{}


\rub{Under det att det allraheligaste Sakramentet bärs fram till altaret sjungs {\color {black} Pange lingua, O Salutaris hostia} eller annan lämplig hymn. Officianten incenserar Sakramentet.\\}

\bigskip

\gregorioscore{mu/pangelinguavatican}

\bigskip

\rub{Officianten läser  en eller flera läsningar ur den Heliga Skrift.\\
Officianten kan hålla en kort homilia. Efter homilian hålls en stunds tyst tillbedjan.\\
Därefter läses en station.
Officianten incenserar Sakramentet medan man sjunger {\color {black} Tantum ergo}. Efter hymnen sjunger eller läser officianten:}

\vspace{10 mm}

\va Panem de c\ae lo præstitísti eis (T. P: allelúia).\\*
\ra Omne delectaméntum in se habéntem (T. P: allelúia).
\bigskip

\gregorioscore{mu/panendecoelo}%


\gregorioscore{mu/omne}

\newpage

\rub{Officianten säger:}

\bigskip

{\large Orémus.}

\smallskip

\ltres{D}{} eus, qui nobis sub sacraménto mirábili passiónis tuæ memóriam
reliquísti, tríbue, qu\'\ae sumus, ita nos Córporis et Sánguinis tui
sacra mystéria venerári, ut redemptiónis tuæ fructum in nobis
iúgiter sentiámus. Qui vivis et regnas in s\'\ae cula sæculórum.

\medskip

\ra Amen.

\vspace{5 mm}

\gregorioscore{./mu/omneoremus}

\medskip

\rub{Efter bönen tar officianten på sig velum, gör knäböjning, tar monstransen (eller ciboriet) och gör med den (det) korstecknet över församlingen, utan att säga något. Församlingen böjer sig i tillbedjan och tar emot den sakramentala välsignelsen.}

\rub{Efter välsignelsen kan följande lovprisning läsas:}

\vspace{5 mm}

{\fontsize{16}{16}\selectfont 

{\color{rubrica}V}älsignad vare {\color{rubrica}G}ud.

\vspace{3 mm}

{\color{rubrica}V}älsignat vare hans heliga namn.

\vspace{3 mm}

{\color{rubrica}V}älsignad vare {\color{rubrica}J}esus {\color{rubrica}K}ristus, sann {\color{rubrica}G}ud och sann människa.

\vspace{3 mm}

{\color{rubrica}V}älsignat vare {\color{rubrica}J}esu namn.

\vspace{3 mm}

{\color{rubrica}V}älsignat vare hans heliga hjärta.

\vspace{3 mm}

{\color{rubrica}V}älsignat vare hans dyrbara blod.

\vspace{3 mm}

{\color{rubrica}V}älsignad vare {\color{rubrica}J}esus i altarets allraheligaste {\color{rubrica}S}akrament.

\vspace{3 mm}

{\color{rubrica}V}älsignad vare den helige {\color{rubrica}A}nde, {\color{rubrica}T}röstaren.

\vspace{3 mm}

{\color{rubrica}V}älsignad vare {\color{rubrica}G}uds {\color{rubrica}M}oder, den heliga {\color{rubrica}J}ungfrun {\color{rubrica}M}aria.

\vspace{3 mm}

{\color{rubrica}V}älsignad vare hennes heliga och obefläckade avlelse.

\vspace{3 mm}

{\color{rubrica}V}älsignad vare hennes ärorika upptagning i himlen.

\vspace{3 mm}

{\color{rubrica}V}älsignat vare {\color{rubrica}M}arias namn, {\color{rubrica}J}ungfru och {\color{rubrica}M}oder.

\vspace{3 mm}

{\color{rubrica}V}älsignad vare den helige {\color{rubrica}J}osef, hennes kyske brudgum.

\vspace{3 mm}

{\color{rubrica}V}älsignad vare {\color{rubrica}G}ud i sina änglar och helgon.\\}




\rub{Efter andakten återställer prästen  som förrättat välsignelsen  Sakramentet till tabernaklet och gör knäböjning. Under tiden kan församlingen läsa eller sjunga någon form av acklamation. Officianten återvänder till sakristian. }

