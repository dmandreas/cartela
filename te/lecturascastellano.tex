\selectlanguage{spanish}
\chapter{TEXTOS PARA PROCLAMAR Y MEDITAR ANTE EL SANTÍSIMO SACRAMENTO}
\markboth{TEXTOS DE LA SAGRADA ESCRITURA}{}

%\vspace{5 mm}


\section{Adviento}

\lect{Lectura del libro de Isaías}{45, 6b-8} 

<<Yo no soy el Señor, y no hay ningún otro, el que 
produce la luz y crea las tinieblas, el que hace 
la paz y crea la desdicha. 

Yo, el Señor, hago 
todo esto. 

Destilad, cielos, el rocío de lo alto, 
derramad, nubes, la justicia, que se abra la 
tierra y germine la salvación, y que, a la vez, 
brote la justicia. Yo, el Señor, lo he creado>>. 

\lect{Lectura del libro de Jeremías   }{ 23, 5-6}


<<Mirad que vienen días --oráculo del Señor-- en 
que suscitaré a David un brote justo, que rija 
como rey y sea prudente, y ejerza el derecho y 
la justicia en la tierra. 

En sus días Judá será 
salvada, e Israel habitará en seguridad, y éste 
será el nombre con que le llamen: El Señor, 
nuestra Justicia>>. 


\lect{Lectura de la carta del apóstol san Pablo a los Filipenses  }{ 4,~4-7}


Hermanos: Alegraos siempre en el Señor; os lo 
repito, alegraos. Que vuestra comprensión sea 
patente a todos los hombres. 

El Señor está 
cerca. No os preocupéis por nada; al contrario: 
en toda oración y súplica, presentad a Dios 
vuestras peticiones con acción de gracias. 

Y la 
paz de Dios que supera todo entendimiento 
custodiará vuestros corazones y vuestros 
pensamientos en Cristo Jesús. 

\newpage
\section{Navidad}

\lect{Lectura del santo Evangelio según san Lucas   }{ 2, 8-12.16}

Había unos pastores por aquellos contornos, 
que dormían al raso y vigilaban por turno su 
rebaño durante la noche. 

De improviso un 
ángel del Señor se les presentó, y la gloria del 
Señor los rodeó de luz. Y se llenaron de un 
gran temor. 

El ángel les dijo: <<No temáis. 
Mirad que vengo a anunciaros una gran 
alegría, que lo será para todo el pueblo: hoy os 
ha nacido, en la ciudad de David, el Salvador, 
que es el Cristo, el Señor; y esto os servirá de 
señal: encontraréis a un niño envuelto en 
pañales y reclinado en un pesebre>>. 

Y vinieron 
presurosos y encontraron a María y a José y al 
niño reclinado en el pesebre. 

\lect{Lectura de la carta del apóstol san Pablo a los 
Gálatas 
  }{ 4, 4-7}

 Hermanos: Al llegar la plenitud de los 
tiempos, envió Dios a su Hijo, nacido de 
mujer, nacido bajo la Ley, para redimir a los 
que estaban bajo la Ley, a fin de que 
recibiésemos la adopción de hijos. 

Y, puesto 
que sois hijos, Dios envió a nuestros corazones 
el Espíritu de su Hijo, que clama: <<!`Abbá, 
Padre!>>  De manera que ya no eres siervo, sino 
hijo; y como eres hijo, también heredero por 
gracia de Dios. 

\lect{Lectura de la carta del apóstol san Pablo a los 
Filipenses 
  }{ 2, 6-11}

Jesucristo, siendo de condición divina, no 
consideró como presa codiciable el ser igual a 
Dios, sino que se anonadó a sí mismo tomando 
la forma de siervo, hecho semejante a los 
hombres; y, mostrándose igual que los demás 
hombres, se humilló a sí mismo haciéndose 
obediente hasta la muerte, y muerte de cruz. 

Y por eso Dios lo exaltó y le otorgó el nombre 
que está sobre todo nombre; para que al 
nombre de Jesús toda rodilla se doble en los 
cielos, en la tierra y en los abismos, y toda 
lengua confiese: <<!`Jesucristo es el Señor!>>, para 
gloria de Dios Padre. 

\lect{Lectura de la carta del apóstol san Pablo a Tito 
  }{ 2, 11-14}

 Queridos hermanos: 
 
 Se ha manifestado la 
gracia de Dios, portadora de salvación para 
todos los hombres, educándonos para que 
renunciemos a la impiedad y a las 
concupiscencias mundanas, y vivamos con 
prudencia, justicia y piedad en este mundo, 
aguardando la esperanza bienaventurada y la 
manifestación de la gloria del gran Dios y 
Salvador nuestro, Jesucristo. 

Él se entregó a sí 
mismo por nosotros para redimirnos de toda 
iniquidad, y para purificar para sí un pueblo 
escogido, celoso por hacer el bien. 

\newpage
\section{Tiempo ordinario}

\lect{Lectura del libro del Éxodo 
  }{  3, 1-7.15}

 En aquellos días Moisés apacentaba el rebaño 
de su suegro Jetró, sacerdote de Madián; solía 
conducirlo al interior del desierto, llegando 
hasta el Horeb, el monte de Dios. 

El ángel del 
Señor se le manifestó en forma de llama de 
fuego en medio de una zarza. Moisés miró: la 
zarza ardía, pero no se consumía. 

Y se dijo 
Moisés: <<Voy a acercarme y comprobar esta 
visión prodigiosa: por qué no se consume la 
zarza>>. 

Vio el Señor que Moisés se acercaba a 
mirar y lo llamó de entre la zarza: <<!`Moisés, 
Moisés! >> Y respondió él: <<Heme aquí>>. Y dijo 
Dios: <<No te acerques aquí; quítate las 
sandalias de los pies, porque el lugar que pisas 
es tierra sagrada>>. Y añadió: <<Yo soy el Dios 
de tu padre, el Dios de Abrahán, el Dios de 
Isaac y el Dios de Jacob>>. 

Moisés se cubrió el 
rostro por temor a contemplar a Dios. Y le dijo 
más: <<Así dirás a los hijos de Israel: El Señor, 
el Dios de vuestros padres, el Dios de 
Abrahán, el Dios de Isaac, el Dios de Jacob, me 
envía a vosotros. Éste es mi nombre para 
siempre; así seré invocado de generación en 
generación>>. 

\lect{Lectura del libro de los Proverbios 
  }{ 9, 1-6}

 La Sabiduría edificó su casa, asentó sus siete 
columnas; inmoló sus víctimas, mezcló su 
vino, preparó su mesa. Envió a sus criadas, y 
clama desde la altura que domina la ciudad: 
<<Quien sea simple, venga acá>>. A los faltos de 
seso les dice: <<Venid, comed de mi pan, y 
bebed del vino que he mezclado. Dejad la 
simpleza y viviréis, avanzad por los caminos 
del discernimiento>>. 

\lect{Lectura del segundo libro de Samuel 
  }{ 6, 1-3.5.15.17-18}

 David reunió de nuevo a lo más selecto de 
Israel, treinta mil hombres. Se levantó y se 
dirigió con todo su ejército hacia Baalá de Judá 
para traer desde allí el arca de Dios, que 
llevaba sobre sí el nombre del Señor de los 
ejércitos que se sienta sobre los querubines. 

Cargaron el arca de Dios sobre una carreta 
nueva y la sacaron de la casa de Abinadab que está en la colina. 
Uzá y Ajió, hijos de 
Abinadab, conducían la carreta. 

David y todo 
Israel iban bailando delante del Señor con todo 
su entusiasmo, cantando con cítaras y arpas, 
con panderos, sistros y címbalos. David y toda 
la casa de Israel trasladaban el arca del Señor 
entre gritos de júbilo y sonar de trompetas. 

Introdujeron el arca del Señor y la colocaron 
en su sitio en medio de la tienda que David 
había mandado levantar. David ofreció ante el 
Señor holocaustos y sacrificios de comunión. 

Y 
cuando terminó la ofrenda del holocausto y de 
los sacrificios de comunión, bendijo al pueblo 
en nombre del Señor de los ejércitos. 

\lect{Lectura del santo Evangelio según san Mateo 
  }{ 6, 25-32}

 En aquel tiempo dijo Jesús a sus discípulos: 
 
<<No estéis preocupados por vuestra vida: qué 
vais a comer; o por vuestro cuerpo: con qué os 
vais a vestir. ?`Es que no vale más la vida que 
el alimento, y el cuerpo más que el vestido? 
Mirad las aves del cielo: no siembran, ni 
siegan, ni almacenan en graneros, y vuestro 
Padre celestial las alimenta. ?`Es que no valéis 
vosotros mucho más que ellas? ?`Quién de 
vosotros, por mucho que cavile, puede añadir 
un solo codo a su estatura? Y sobre el vestir, 
?`por qué os preocupáis? Fijaos en los lirios del 
campo, cómo crecen; no se fatigan ni hilan, y 
yo os digo que ni Salomón en toda su gloria 
pudo vestirse como uno de ellos. Y si a la 
hierba del campo, que hoy es y mañana se 
echa al horno, Dios la viste así, ?`cuánto más a 
vosotros, hombres de poca fe? 
Así pues, no 
andéis preocupados diciendo: <<?`qué vamos a 
comer, qué vamos a beber, con qué nos vamos 
a vestir?>> 

Por todas esas cosas se afanan los 
paganos. Bien sabe vuestro Padre celestial que 
de todo eso estáis necesitados>>. 

\lect{Lectura del santo Evangelio según san Lucas 
  }{ 22, 14-20}

 Llegada la hora, se puso a la mesa y los 
apóstoles con él. Y les dijo: <<Ardientemente he 
deseado comer esta Pascua con vosotros, antes 
de padecer, porque os digo que no la volveré a 
comer hasta que tenga su cumplimiento en el 
Reino de Dios>>. 

Y tomando el cáliz, dio gracias y dijo: <<Tomadlo y distribuidlo entre vosotros; 
pues os digo que a partir de ahora no beberé 
del fruto de la vid hasta que venga el Reino de 
Dios>>. 

Y tomando pan, dio gracias, lo partió y 
se lo dio diciendo: <<Esto es mi cuerpo, que es 
entregado por vosotros. Haced esto en 
memoria mía>>. 

Y del mismo modo el cáliz, 
después de haber cenado, diciendo: <<Este cáliz 
es la nueva alianza en mi sangre, que es 
derramada por vosotros>>. 

\lect{Lectura del santo Evangelio según san Juan 
  }{ 6, 51-58}

 En aquel tiempo dijo Jesús a los Judíos: <<Yo 
soy el pan vivo que ha bajado del cielo. Si 
alguno come este pan vivirá eternamente; y el 
pan que yo daré es mi carne para la vida del 
mundo>>. 

Los judíos se pusieron a discutir 
entre ellos: <<?`Cómo puede éste darnos a comer 
su carne?>> 

Jesús les dijo: <<En verdad, en 
verdad os digo que si no coméis la carne del 
Hijo del Hombre y no bebéis su sangre, no 
tendréis vida en vosotros. El que come mi 
carne y bebe mi sangre tiene vida eterna, y yo 
le resucitaré en el último día. Porque mi carne 
es verdadera comida y mi sangre es verdadera 
bebida. El que come mi carne y bebe mi sangre 
permanece en mí y yo en él. Igual que el Padre 
que me envió vive y yo vivo por el Padre, así, 
aquel que me come vivirá por mí. Éste es el 
pan que ha bajado del cielo, no como el que 
comieron los padres y murieron: quien come 
este pan vivirá eternamente>>. 

\lect{Lectura del santo Evangelio según san Juan 
  }{ 14, 23-27}

 En aquel tiempo dijo Jesús a sus discípulos: 
<<Si alguno me ama, guardará mi palabra, y mi 
Padre le amará, y vendremos a él y haremos 
morada en él. El que no me ama, no guarda 
mis palabras; y la palabra que escucháis no es 
mía, sino del Padre que me ha enviado. 

Os he 
hablado de todo esto estando con vosotros; 
pero el Paráclito, el Espíritu Santo que el Padre 
enviará en mi nombre, Él os enseñará todo y 
os recordará todas las cosas que os he dicho. 
La paz os dejo, mi paz os doy; no os la doy
como la da el mundo. No se turbe vuestro 
corazón ni se acobarde>>. 

\lect{Lectura de la carta del apóstol san Pablo a los 
Corintios 
  }{ 10, 16-17}

 Hermanos: 
El cáliz de bendición que bendecimos ?`no es la comunión de la sangre 
de Cristo? El pan que partimos ?`no es la 
comunión del Cuerpo de Cristo? Puesto que el 
pan es uno, muchos somos un solo cuerpo, 
porque todos participamos de un solo pan. 

\lect{Lectura de la carta del apóstol san Pablo a los 
Corintios 
  }{ 11, 23-26}

 Hermanos: Yo recibí del Señor lo que también 
os transmití: que el Señor Jesús, la noche en 
que fue entregado, tomó pan, y dando gracias, 
lo partió y dijo:

 <<Esto es mi cuerpo, que se da 
por vosotros; haced esto en conmemoración 
mía>>. 

Y de la misma manera, después de 
cenar, tomó el cáliz, diciendo: 

<<Este cáliz es la 
nueva alianza en mi sangre; cuantas veces lo 
bebáis, hacedlo en conmemoración mía>>. 

Porque cada vez que coméis este pan y bebéis 
este cáliz, anunciáis la muerte del Señor, hasta 
que venga. 

\lect{Lectura de la primera carta del apóstol san Juan 
  }{ 5, 4-8}

 Queridos hermanos: Todo el que ha nacido de 
Dios, vence al mundo. Y ésta es la victoria que 
ha vencido al mundo: nuestra fe.

?`Quién es el 
que vence al mundo sino el que cree que Jesús 
es el Hijo de Dios? Éste es el que vino por el 
agua y por la sangre: Jesucristo. 

No solamente 
con el agua, sino con el agua y con la sangre. Y 
es el Espíritu quien da testimonio, porque el 
Espíritu es la verdad. Pues son tres los que dan 
testimonio: el Espíritu, el agua y la sangre, y 
los tres coinciden en lo mismo. 

\lect{Lectura del libro del Apocalipsis 
  }{ 7,~9-12}


Yo, Juan, vi una gran multitud que nadie 
podría contar, de todas las naciones, tribus, 
pueblos y lenguas, de pie ante el trono y ante 
el Cordero, vestidos con túnicas blancas, y con 
palmas en las manos, que gritaban con fuerte 
voz: 

<<!`La salvación viene de nuestro Dios, que 
se sienta sobre el trono, y del Cordero! >>

Y 
todos los ángeles estaban de pie alrededor del 
trono, de los ancianos y de los cuatro seres 
vivos, y cayeron sobre sus rostros ante el trono 
y adoraron a Dios, diciendo:

 <<Amén. La 
bendición, la gloria, la sabiduría, la acción de 
gracias, el honor, el poder y la fortaleza 
pertenecen a nuestro Dios por los siglos de los 
siglos. Amén>>. 

\lect{Lectura del libro del Apocalipsis 
  }{ 21, 1-4}

 
Yo, Juan, vi un cielo nuevo y una tierra nueva, 
pues el primer cielo y la primera tierra 
desaparecieron, y el mar ya no existe. Vi 
también la ciudad santa, la nueva Jerusalén, 
que bajaba del cielo de parte de Dios, ataviada 
como una novia que se engalana para su 
esposo. 

Y oí una fuerte voz procedente del 
trono que decía: 

<<Ésta es la morada de Dios 
con los hombres: Habitará con ellos y ellos 
serán su pueblo, y Dios, habitando realmente 
en medio de ellos, será su Dios. Y enjugará 
toda lágrima de sus ojos; y no habrá ya 
muerte, ni llanto, ni lamento, ni dolor, porque 
todo lo anterior ya pasó>>. 

\newpage
\section{Cuaresma}

\lect{Lectura del libro del Deuteronomio 
  }{ 8, 2-3.14b-16a }

 En aquellos días habló Moisés al pueblo, 
diciendo: 

<<Debes recordar todo el camino que 
el Señor, tu Dios, te ha hecho recorrer por el 
desierto durante estos cuarenta años, para 
hacerte humilde, para probarte y conocer lo 
que hay en tu corazón, si guardas o no sus 
mandamientos. 

Te humilló y te hizo pasar 
hambre. Luego te alimentó con el maná, que 
desconocíais tú y tus padres, para enseñarte 
que no sólo de pan vive el hombre, sino de 
todo lo que sale de la boca del Señor.

 No te 
olvides del Señor, tu Dios, que te sacó del país 
de Egipto, de la casa de la esclavitud, el que te 
ha conducido por el desierto grande y terrible, 
con serpientes venenosas y alacranes, por un 
secarral en el que no hay agua. 

Él es el que 
hizo brotar para ti agua de la roca de pedernal; 
el que te alimentó en el desierto con el maná, 
que no habían conocido tus padres>>. 

\lect{Lectura del primer libro de los Reyes 
  }{ 19, 4-8}

Elías continuó por el desierto 
una jornada de camino, y, al final, se sentó 
bajo una retama y se deseó la muerte 
diciendo:

<<Ya es demasiado, Señor, toma mi 
vida, pues yo no soy mejor que mis padres>>. 
Se echó y se quedó dormido debajo de la 
retama. 

De pronto, un ángel le tocó y le dijo: 
<<Levántate y come>>. Miró a su cabecera y 
había una torta asada y un jarro de agua. Él 
comió y bebió; luego se volvió a echar. El 
ángel del Señor volvió a tocarle por segunda 
vez y le dijo: <<Levántate y come, porque te 
queda un camino demasiado largo>>. 

Se 
levantó, comió y bebió; y con las fuerzas de 
aquella comida caminó cuarenta días y 
cuarenta noches hasta el Horeb, el monte de 
Dios.

\lect{Lectura del santo Evangelio según san Juan   }{ 6, 26-29}

 En aquel tiempo, Jesús les respondió: <<En 
verdad, en verdad os digo que vosotros me 
buscáis no por haber visto los signos, sino 
porque habéis comido los panes y os habéis 
saciado. Obrad no por el alimento que se 
consume, sino por el que perdura hasta la vida 
eterna, el que os dará el Hijo del Hombre, pues 
a éste lo confirmó Dios Padre con su sello>>. 


Ellos le preguntaron: <<?`Qué debemos hacer 
para realizar las obras de Dios?>> 


Jesús les 
respondió: <<Ésta es la obra de Dios: que creáis 
en quien Él ha enviado>>. 

\lect{Lectura del santo Evangelio según san Juan 
  }{ 6, 30-35}

 En aquel tiempo, dijo la gente a Jesús: <<?`Y qué 
signo haces tú, para que lo veamos y te 
creamos? ?`Qué obras realizas tú? Nuestros 
padres comieron en el desierto el maná, como 
está escrito: ``Les dio a comer pan del cielo''>>.

Les respondió Jesús: <<En verdad, en verdad os 
digo que Moisés no os dio el pan del cielo, sino 
que mi Padre os da el verdadero pan del cielo. 
Porque el pan de Dios es el que ha bajado del 
cielo y da la vida al mundo>>. <<Señor, danos 
siempre de este pan>>, le dijeron ellos. 

Jesús les 
respondió: <<Yo soy el pan de vida; el que viene 
a mí no tendrá hambre, y el que cree en mí no 
tendrá nunca sed>>. 

\newpage
\section{Tiempo pascual}

\lect{Lectura del santo Evangelio según san Lucas 
  }{ 24, 13-16.28-34}

 Dos de sus discípulos se dirigían a una aldea 
llamada Emaús, que distaba de Jerusalén 
sesenta estadios. Iban conversando entre sí de 
todo lo que había acontecido. Y mientras 
comentaban y discutían, el propio Jesús se 
acercó y se puso a caminar con ellos, aunque 
sus ojos eran incapaces de reconocerle. 

Llegaron cerca de la aldea adonde iban, y él 
hizo ademán de continuar adelante. Pero le 
retuvieron diciéndole: <<Quédate con nosotros, 
porque se hace tarde y está ya anocheciendo>>. 
Y entró para quedarse con ellos. 

Y cuando 
estaban juntos a la mesa tomó el pan, lo 
bendijo, lo partió y se lo dio. Entonces se les 
abrieron los ojos y le reconocieron, pero él 
desapareció de su presencia. 

Y se dijeron uno a 
otro: <<?`No es verdad que ardía nuestro 
corazón dentro de nosotros, mientras nos 
hablaba por el camino y nos explicaba las 
Escrituras? >> 

Y, al instante, se levantaron y 
regresaron a Jerusalén, y encontraron reunidos 
a los once y a los que estaban con ellos, que 
decían: <<El Señor ha resucitado realmente y se 
ha aparecido a Simón>>. 

\lect{Lectura del santo Evangelio según san Juan 
  }{ 20, 19-21}

 Al atardecer de aquel día, el siguiente al 
sábado, con las puertas del lugar donde se 
habían reunido los discípulos cerradas por 
miedo a los judíos, vino Jesús, se presentó en 
medio de ellos y les dijo: <<La paz esté con 
vosotros>>. Y dicho esto, les mostró las manos y 
el costado. 

Al ver al Señor, los discípulos se 
alegraron. Les repitió: <<La paz esté con 
vosotros. Como el Padre me envió, así os envío 
yo>>. 

\lect{Lectura del santo Evangelio según san Juan 
  }{ 20, 26-29}

 A los ocho días, estaban otra vez dentro sus 
discípulos y Tomás con ellos. Aunque estaban 
las puertas cerradas, vino Jesús, se presentó en 
medio y dijo: <<La paz esté con vosotros>>. 

Después le dijo a Tomás: <<Trae aquí tu dedo y 
mira mis manos, y trae tu mano y métela en mi 
costado, y no seas incrédulo, sino creyente>>. 

Respondió Tomás y le dijo: <<!`Señor mío y Dios 
mío! >> 

Jesús contestó: <<Porque me has visto 
has creído; bienaventurados los que sin haber 
visto hayan creído>>. 

\lect{Lectura del libro de los Hechos de los apóstoles 
  }{ 2, 42-47}

 Perseveraban asiduamente en la doctrina de 
los apóstoles y en la comunión, en la fracción 
del pan y en las oraciones. El temor sobrecogía 
a todos, y por medio de los apóstoles se 
realizaban muchos prodigios y señales. 

Todos 
los creyentes estaban unidos y tenían todas las 
cosas en común. Vendían las posesiones y los 
bienes y los repartían entre todos, según las 
necesidades de cada uno. 

Todos los días 
acudían al Templo con un mismo espíritu, 
partían el pan en las casas y comían juntos con 
alegría y sencillez de corazón, alabando a Dios 
y gozando del favor de todo el pueblo. Todos 
los días el Señor incorporaba a los que habían 
de salvarse. 

\lect{Lectura del libro de los Hechos de los apóstoles 
  }{ 2, 42; 4, 32-33}

 Perseveraban asiduamente en la doctrina de 
los apóstoles y en la comunión, en la fracción 
del pan y en las oraciones. 

Y la multitud de los 
creyentes tenía un solo corazón y una sola 
alma, y nadie consideraba como suyo lo que 
poseía, sino que compartían todas las cosas. 

Con gran poder los apóstoles daban 
testimonio de la resurrección del Señor Jesús; y 
en todos ellos había abundancia de gracia. 

\newpage
\section{Solemnidades y fiestas}

\fest{1 de enero}{Santa María, madre de Dios}

\lect{Lectura del santo Evangelio según san Lucas 
  }{ 2, 16-20}

 En aquel tiempo: Vinieron presurosos a Belén 
y encontraron a María y a José y al niño 
reclinado en el pesebre. Al verlo, reconocieron 
las cosas que les habían sido anunciadas sobre 
este niño.

 Y todos los que lo oyeron se 
maravillaron de cuanto los pastores les habían 
dicho. 

María guardaba todas estas cosas 
ponderándolas en su corazón. 

Y los pastores 
regresaron, glorificando y alabando a Dios por 
todo lo que habían oído y visto, según les fue 
dicho. 


\fest{6 de enero}{Epifanía del Señor}


\lect{Lectura del santo Evangelio según san Mateo 
  }{ 2, 1-2.7-11}

 Después de nacer Jesús en Belén de Judá en 
tiempos del rey Herodes, unos Magos llegaron 
de Oriente a Jerusalén preguntando:

 <<?`Dónde 
está el Rey de los Judíos que ha nacido? 
Porque vimos su estrella en el Oriente y hemos 
venido a adorarle>>. 

Entonces, Herodes, 
llamando en secreto a los Magos, se informó 
cuidadosamente por ellos del tiempo en que 
había aparecido la estrella; y les envió a Belén, 
diciéndoles: 

<<Id e informaos bien acerca del 
niño; y cuando lo encontréis, avisadme para 
que también yo vaya a adorarle>>. 

Ellos, 
después de oír al rey, se pusieron en marcha. Y 
entonces, la estrella que habían visto en el 
Oriente se colocó delante de ellos, hasta 
pararse sobre el sitio donde estaba el niño. Al 
ver la estrella se llenaron de inmensa alegría. Y 
entrando en la casa, vieron al niño con María, 
su madre, y postrándose le adoraron; luego, 
abrieron sus cofres y le ofrecieron presentes, 
oro, incienso y mirra. 


\fest{19 de marzo }{San José}


\lect{Lectura del libro del Génesis 
  }{ 41, 55-57}

 En aquellos días, llegó también el hambre a 
todo el país de Egipto, y el pueblo clamó al 
faraón pidiendo pan. El faraón dijo a todos los 
egipcios: 

<<Id a José, y haced lo que él os diga>>. 

Reinaba el hambre sobre toda la faz de la 
tierra, y entonces José abrió todos los graneros 
y vendió grano a los egipcios mientras 
arreciaba el hambre en el país de Egipto. De 
todos los países venían a Egipto a comprar 
grano a José, porque el hambre arreciaba en 
toda la tierra. 



\fest{25 de marzo}{Anunciación del Señor}

\lect{Lectura del santo Evangelio según san Juan 
  }{ 1, 1-5.9-14}

 En el principio existía el Verbo, y el Verbo 
estaba junto a Dios, y el Verbo era Dios. Él 
estaba en el principio junto a Dios. 

Todo se 
hizo por él, y sin él no se hizo nada de cuanto 
ha sido hecho. En él estaba la vida, y la vida 
era la luz de los hombres. Y la luz brilla en las 
tinieblas, y las tinieblas no la recibieron.

 El 
Verbo era la luz verdadera, que ilumina a todo 
hombre, que viene a este mundo. En el mundo 
estaba, y el mundo se hizo por él, y el mundo 
no le conoció. 

Vino a los suyos, y los suyos no 
le recibieron. Pero a cuantos le recibieron les 
dio la potestad de ser hijos de Dios, a los que 
creen en su nombre, que no han nacido de la 
sangre, ni de la voluntad de la carne, ni del 
querer del hombre, sino de Dios. 

Y el Verbo se 
hizo carne, y habitó entre nosotros, y hemos 
visto su gloria, gloria como de Unigénito del 
Padre, lleno de gracia y de verdad. 

\fest{24 de junio}{Natividad de San Juan Bautista}


\lect{Lectura del santo Evangelio según san Juan 
  }{ 1, 35-39}

 En aquel tiempo estaban Juan y dos de sus 
discípulos y, fijándose en Jesús que pasaba, 
dijo: <<Éste es el Cordero de Dios>>. 

 Los dos 
discípulos, al oírle hablar así, siguieron a Jesús. 
Se volvió Jesús y, viendo que le seguían, les 
preguntó: <<?`Qué buscáis? >> Ellos le dijeron: 
<<Rabbí -que significa: <<Maestro>>-, ?`dónde 
vives?>>. Les respondió: <<Venid y veréis>>.  


Fueron y vieron dónde vivía, y se quedaron 
con él aquel día. Era más o menos la hora 
décima. 


\fest{29 de junio }{San Pedro y San Pablo}


\lect{Lectura del santo Evangelio según san Mateo 
  }{ 28, 16-20}

 En aquel tiempo los once discípulos 
marcharon a Galilea, al monte que Jesús les 
había indicado y en cuanto le vieron le 
adoraron; pero otros dudaron.  

Y Jesús se 
acercó y les dijo: <<Se me ha dado toda 
potestad en el cielo y en la tierra. Id, pues, y 
haced discípulos a todos los pueblos, 
bautizándoles en el nombre del Padre y del 
Hijo y del Espíritu Santo; y enseñándoles a 
guardar todo cuanto os he mandado. Y sabed 
que yo estoy con vosotros todos los días hasta 
el fin del mundo>>. 



\fest{15 de agosto }{ Asunción de la Virgen María}


\lect{Lectura del libro del Apocalipsis 
  }{ 21, 1-4}

 Yo, Juan, vi un cielo nuevo y una tierra nueva, 
pues el primer cielo y la primera tierra 
desaparecieron, y el mar ya no existe. Vi 
también la ciudad santa, la nueva Jerusalén, 
que bajaba del cielo de parte de Dios, ataviada 
como una novia que se engalana para su 
esposo. 

 Y oí una fuerte voz procedente del 
trono que decía: 

 <<Ésta es la morada de Dios 
con los hombres: habitará con ellos y ellos 
serán su pueblo, y Dios, habitando realmente 
en medio de ellos, será su Dios. Y enjugará 
toda lágrima de sus ojos; y no habrá ya 
muerte, ni llanto, ni lamento, ni dolor, porque 
todo lo anterior ya pasó>>. 


\fest{14 de septiembre }{ Exaltación de la Santa Cruz}


\lect{Lectura de la carta del apóstol san Pablo a los 
Filipenses 
  }{ 2, 5-11}

 Hermanos: Tened entre vosotros los mismos 
sentimientos que tuvo Cristo Jesús, el cual, 
siendo de condición divina, no consideró 
como presa codiciable el ser igual a Dios, sino 
que se anonadó a sí mismo tomando la forma 
de siervo, hecho semejante a los hombres; y, 
mostrándose igual que los demás hombres, se 
humilló a sí mismo haciéndose obediente
hasta la muerte, y muerte de cruz. 

 Y por eso 
Dios lo exaltó y le otorgó el nombre que está 
sobre todo nombre; para que al nombre de 
Jesús toda rodilla se doble en los cielos, en la 
tierra y en los abismos, y toda lengua confiese: 
<<!`Jesucristo es el Señor!>>, para gloria de Dios 
Padre. 


\fest{29 de septiembre }{ San Miguel, San Gabriel y San Rafael}


\lect{Lectura del libro del Apocalipsis }{ 8, 2-4}

 
Yo, Juan, vi a los siete ángeles que están de pie delante de Dios. Les entregaron siete 
trompetas. Vino otro ángel y se quedó en pie 
junto al altar con un incensario de oro. Le 
entregaron muchos perfumes para que los 
ofreciera, con las oraciones de todos los santos, 
sobre el altar de oro que está ante el trono. Y 
ascendió el humo de los perfumes, con las 
oraciones de los santos, desde la mano del 
ángel hasta la presencia de Dios. 


\fest{1 de noviembre }{ Todos los santos }


\lect{Lectura del libro del Apocalipsis 
  }{ 7, 9-12}

 Yo, Juan, vi una gran multitud que nadie 
podía contar, de todas las naciones, tribus, 
pueblos y lenguas, de pie ante el trono y ante 
el Cordero, vestidos con túnicas blancas, y con 
palmas en las manos, que gritaban con fuerte 
voz:  

<<!`La salvación viene de nuestro Dios, que 
se sienta sobre el trono, y del Cordero! >>  

Y 
todos los ángeles estaban de pie alrededor del 
trono, de los ancianos y de los cuatro seres 
vivos, y cayeron sobre sus rostros ante el trono 
y adoraron a Dios, diciendo:  

<<Amén. La 
bendición, la gloria, la sabiduría, la acción de 
gracias, el honor, el poder y la fortaleza 
pertenecen a nuestro Dios por los siglos de los 
siglos. Amén>>. 

\fest{8 de diciembre }{ Inmaculada Concepción}


\lect{Lectura de la profecía de Zacarías 
  }{ 2, 14-15}

 <<Grita de gozo y alégrate, hija de Sión, porque 
vengo a habitar dentro de ti -oráculo del 
Señor. Aquel día seguirán al Señor muchas 
naciones y serán pueblo mío. Yo habitaré en 
medio de ti, y sabrás que el Señor de los 
ejércitos a ti me envió>>. 


\lect{Lectura del libro de los Salmos 
  }{ 44 (45), 11-18}

 Escucha, hija, y mira, presta tu oído, olvida tu 
pueblo y la casa de tu padre: y el rey se 
prendará de tu belleza; él es tu señor, inclínate 
ante él. La hija de Tiro viene con presentes, los 
pueblos más ricos buscan tu favor. Radiante 
de gloria, la hija del rey enjoyada de brocados 
de oro es su vestido, con bordados de colores, 
es conducida ante el rey. Vírgenes, sus damas, 
forman su séquito, son conducidas ante ti; son 
conducidas en medio de alegría y regocijo; 
entran en el palacio del rey. En lugar de tus 
padres estarán tus hijos; los constituirás 
príncipes por toda la tierra. Haré memorable 
tu nombre en todas las generaciones; por esto, 
te alabarán los pueblos por los siglos de los 
siglos. 

\lect{Lectura de la carta del apóstol san Pablo a los 
Romanos 
  }{ 8, 28-30}

 Hermanos: Sabemos que todas las cosas 
cooperan para el bien de los que aman a Dios, 
de los que son llamados según su designio. 
Porque a los que de antemano eligió, también 
predestinó para que lleguen a ser conformes a 
la imagen de su Hijo, a fin de que él sea 
primogénito entre muchos hermanos. Y a los 
que predestinó también los llamó, y a los que 
llamó también los justificó, y a los que justificó 
también los glorificó. 


\fest{25 de diciembre }{Natividad del Señor}


\lect{Lectura del santo Evangelio según san Juan 
  }{ 1, 1-5.9-14}

 En el principio existía el Verbo, y el Verbo 
estaba junto a Dios, y el Verbo era Dios. Él 
estaba en el principio junto a Dios. Todo se 
hizo por él, y sin él no se hizo nada de cuanto 
ha sido hecho. En él estaba la vida, y la vida 
era la luz de los hombres. Y la luz brilla en las 
tinieblas, y las tinieblas no la recibieron.  

El 
Verbo era la luz verdadera, que ilumina a todo 
hombre, que viene a este mundo. En el mundo 
estaba, y el mundo se hizo por él, y el mundo 
no le conoció. Vino a los suyos, y los suyos no 
le recibieron. Pero a cuantos le recibieron les 
dio la potestad de ser hijos de Dios, a los que 
creen en su nombre, que no han nacido de la 
sangre, ni de la voluntad de la carne, ni del 
querer del hombre, sino de Dios. 

 Y el Verbo se 
hizo carne, y habitó entre nosotros, y hemos 
visto su gloria, gloria como de Unigénito del 
Padre, lleno de gracia y de verdad. 


\fest{27 de diciembre }{ San Juan}


\lect{Lectura de la primera carta del apóstol san Juan 
  }{ 4, 7-10}

 Queridísimos: amémonos unos a otros, porque 
el amor procede de Dios, y todo el que ama ha 
nacido de Dios, y conoce a Dios. El que no ama 
no ha llegado a conocer a Dios, porque Dios es 
amor. 

 En esto se manifestó entre nosotros el 
amor de Dios: en que Dios envió a su Hijo 
Unigénito al mundo para que recibiéramos por 
él la vida.  

En esto consiste el amor: no en que nosotros hayamos amado a Dios, sino en que 
Él nos amó y envió a su Hijo como víctima 
propiciatoria por nuestros pecados. 

\newpage

\section{Celebraciones movibles}

\subsection{Resurrección del Señor}


\lect{Lectura del santo Evangelio según san Juan 
  }{ 20, 19-21}

 Al atardecer de aquel día, el siguiente al 
sábado, con las puertas del lugar donde se 
habían reunido los discípulos cerradas por 
miedo a los judíos, vino Jesús, se presentó en 
medio de ellos y les dijo: 

<<La paz esté con vosotros>>. 

Y dicho esto, les mostró las manos y 
el costado. 

Al ver al Señor, los discípulos se 
alegraron. 

Les repitió: <<La paz esté con 
vosotros. Como el Padre me envió, así os envío 
yo>>. 


\subsection{Ascensión del Señor }


\lect{Lectura del santo Evangelio según san Juan 
  }{ 14, 18-21.27-28}

 En aquel tiempo, dijo Jesús a sus discípulos: 
<<No os dejaré huérfanos, yo volveré a 
vosotros. Todavía un poco más y el mundo ya 
no me verá, pero vosotros me veréis porque yo 
vivo y también vosotros viviréis. Ese día 
conoceréis que yo estoy en el Padre, y vosotros 
en mí y yo en vosotros. 

El que acepta mis 
mandamientos y los guarda, ése es el que me 
ama. Y el que me ama será amado por mi 
Padre, y yo le amaré y yo mismo me 
manifestaré a él. 

La paz os dejo, mi paz os doy; 
no os la doy como la da el mundo. No se turbe 
vuestro corazón ni se acobarde. 

Habéis 
escuchado que os he dicho: Me voy y vuelvo a 
vosotros. Si me amarais os alegraríais de que 
vaya al Padre, porque el Padre es mayor que 
yo>>. 


\subsection{Pentecostés }


\lect{Lectura del santo Evangelio según san Juan 
  }{ 14, 23-27}

 En aquel tiempo dijo Jesús a sus discípulos: 
<<Si alguno me ama, guardará mi palabra, y mi 
Padre le amará, y vendremos a él y haremos 
morada en él. El que no me ama, no guarda 
mis palabras; y la palabra que escucháis no es 
mía, sino del Padre que me ha enviado. 

Os he 
hablado de todo esto estando con vosotros; 
pero el Paráclito, el Espíritu Santo que el Padre 
enviará en mi nombre, Él os enseñará todo y 
os recordará todas las cosas que os he dicho. 

La paz os dejo, mi paz os doy; no os la doy 
como la da el mundo. No se turbe vuestro 
corazón ni se acobarde>>. 


\subsection{Santísima Trinidad }


\lect{Lectura del santo Evangelio según san Lucas   }{ 10, 21-24}

 En aquel tiempo lleno de gozo en el Espíritu 
Santo, dijo Jesús: 

<<Yo te alabo, Padre, Señor 
del cielo y de la tierra, porque has ocultado 
estas cosas a los sabios y prudentes y las has 
revelado a los pequeños. Sí, Padre, porque así 
te ha parecido bien. Todo me lo ha entregado 
mi Padre, y nadie conoce quién es el Hijo sino 
el Padre, ni quién es el Padre sino el Hijo, y 
aquel a quien el Hijo quiera revelarlo>>. 

Y 
volviéndose hacia los discípulos les dijo 
aparte: <<Bienaventurados los ojos que ven lo 
que estáis viendo. Pues os aseguro que 
muchos profetas y reyes quisieron ver lo que 
vosotros estáis viendo y no lo vieron; y oír lo 
que estáis oyendo y no lo oyeron>>. 


\subsection{Santísimo Cuerpo y Sangre de Cristo}


\lect{Lectura del libro de la Sabiduría 
  }{  16, 20-21.25-26}

 A tu pueblo, lo alimentaste con manjar de 
ángeles, y les diste pan del cielo, preparado sin 
trabajo, que producía completo deleite, apto 
para todos los gustos. 

Esta sustancia tuya 
mostraba tu dulzura con los hijos, pues servía 
al deseo del que la recibía, se convertía en lo 
que cada uno prefería.  

Por eso, también 
entonces, adoptando todas las formas, servía a 
tu liberalidad, que todo lo nutre, según el 
querer de los necesitados; para que tus hijos 
que amabas, Señor, aprendieran que no son los 
diversos frutos lo que alimenta al hombre, sino 
que es tu palabra la que mantiene a los que 
creen en Ti. 


\subsection*{Santísimo Cuerpo y Sangre de Cristo (Procesión)}

\lect{Lectura del primer libro del las Crónicas 
  }{  15, 3-4.15-16; 16, 1-2}

 En aquellos días convocó David en asamblea 
a todo Israel en Jerusalén para subir el arca del 
Señor al lugar que le había preparado. Reunió 
también a los hijos de Aarón y a los levitas. Y 
los levitas trasladaron el arca de Dios 
poniendo los varales sobre sus hombros, como 
lo había ordenado Moisés, según la palabra del 
Señor. 

 David dijo a los jefes de los levitas que 
dispusieran a sus hermanos los cantores, con 
instrumentos musicales, arpas, cítaras y 
címbalos, para que los hiciesen resonar con 
fuerza en señal de júbilo.  

Así pues, 
introdujeron el arca de Dios y la colocaron en 
medio de la tienda que David había hecho 
levantar; y ofrecieron ante Dios holocaustos y 
sacrificios de comunión. 

 Cuando terminó de 
ofrecer los holocaustos y los sacrificios de 
comunión, David bendijo al pueblo en nombre 
del Señor.

\subsection{Sagrado Corazón de Jesús}

\lect{Lectura del santo Evangelio según san Juan 
  }{ 1, 31-37}

 Como era la Parasceve, para que no se 
quedaran los cuerpos en la cruz el sábado, 
porque aquel sábado era un día grande, los 
judíos rogaron a Pilato que les rompieran las 
piernas y los retirasen.  

Vinieron los soldados y 
rompieron las piernas al primero y al otro que 
había sido crucificado con él. Pero cuando 
llegaron a Jesús, al verle ya muerto, no le 
quebraron las piernas, sino que uno de los 
soldados le abrió el costado con la lanza. Y al 
instante brotó sangre y agua. 

 El que lo vio da 
testimonio, y su testimonio es verdadero; y él 
sabe que dice la verdad para que también 
vosotros creáis. 

 Esto ocurrió para que se 
cumpliera la Escritura: <<No le quebrantarán ni 
un hueso>>.  

Y también otro pasaje de la 
Escritura dice: <<Mirarán al que traspasaron>>. 

\subsection{Jesucristo, rey del universo}


\lect{Lectura del libro del Apocalipsis 
  }{ 5, 11-14}

 Yo, Juan, vi y oí un clamor de muchos ángeles 
que rodeaban el trono, a los seres vivos y a los 
ancianos. Su número era de miríadas de 
miríadas y millares de millares, que aclamaban 
con gran voz: 

 <<Digno es el Cordero inmolado 
de recibir el poder, la riqueza, la sabiduría, la 
fuerza, el honor, la gloria y la alabanza>>. 

 Y a 
toda criatura que existe en el cielo y en la 
tierra, por debajo de la tierra y en el mar, y a 
todo cuanto existe en ellos, les oí decir:  

<<Al 
que está sentado en el trono y al Cordero, la 
alabanza, el honor, la gloria y el poder por los 
siglos de los siglos>>.  

Y los cuatro seres vivos 
respondían: <<Amén>>.  

Y los ancianos se 
postraron y adoraron. 

\subsection{La sagrada familia}


\lect{Lectura del santo Evangelio según san Lucas 
  }{ 2, 4-7}

 En aquellos días,  José, como era de la casa y 
familia de David, subió desde Nazaret, ciudad 
de Galilea, a la ciudad de David llamada 
Belén, en Judea, para empadronarse con 
María, su esposa, que estaba encinta.  

Y cuando 
ellos se encontraban allí, le llegó la hora del 
parto, y dio a luz a su hijo primogénito; lo 
envolvió en pañales y lo recostó en un pesebre, 
porque no había lugar para ellos en el 
aposento. 


\newpage

\section[Aniversarios del Opus Dei]{Aniversarios del Opus Dey y de San Josemaría}

\fest{9 de enero}{Nacimiento de San Josemaría }

\lect{Lectura del libro de Jeremías 
  }{ 1, 4-10}

 Recibí esta palabra del Señor: <<Antes de 
plasmarte en el seno materno, te conocí, antes 
de que salieras de las entrañas, te consagré, te 
puse como profeta de las naciones>>.  

Respondí: 
<<!`Ay, Señor Dios mío! Si no sé hablar, que soy 
muy joven>>.  

El Señor me contestó: <<No digas 
que “soy muy joven”, porque allá donde te 
envíe, irás, y todo cuanto te ordene, lo dirás. 
No les tengas miedo, que Yo estoy contigo 
para librarte>>, oráculo del Señor. El Señor 
extendió su mano, tocó mi boca, y me dijo: 
<<Pongo mis palabras en tu boca. Mira, hoy te 
he puesto sobre las naciones y los reinos, para 
arrancar y abatir, para destruir y arruinar, 
para edificar y plantar>>. 


\fest{14 de febrero}{Fundación de la labor apostólica con mujeres y de la Sociedad Sacerdotal de la Santa Cruz }


\lect{Lectura de la primera carta del apóstol san Pedro 
  }{ 2, 2-5.9-10}

 Queridos hermanos: Como niños recién 
nacidos que ansían la leche, apeteced vosotros 
la leche espiritual, no adulterada, para que con 
ella crezcáis hacia la salvación, si es que habéis 
gustado qué bueno es el Señor.  

Acercándoos a 
él, piedra viva, desechada por los hombres, 
pero escogida y preciosa delante de Dios, 
también vosotros -como piedras vivas- sois 
edificados como edificio espiritual para un 
sacerdocio santo, con el fin de ofrecer 
sacrificios espirituales, agradables a Dios por 
medio de Jesucristo.  

Pero vosotros sois linaje 
escogido, sacerdocio real, nación santa, pueblo 
adquirido en propiedad, para que pregonéis 
las maravillas de Aquel que os llamó de las 
tinieblas a su admirable luz: los que un tiempo 
no erais pueblo, ahora sois pueblo de Dios, los 
que antes no habíais alcanzado misericordia, 
ahora habéis alcanzado misericordia. 

\fest{28 de marzo}{Ordenación sacerdotal de San Josemaría}


\lect{Lectura de la carta a los Hebreos 
  }{ 5, 1-10}

 Todo sumo sacerdote, escogido entre los 
hombres, está constituido en favor de los 
hombres en lo que se refiere a Dios, para 
ofrecer dones y sacrificios por los pecados; y 
puede compadecerse de los ignorantes y 
extraviados, ya que él mismo está rodeado de 
debilidad, y a causa de ella debe ofrecer 
expiación por los pecados, tanto por los del 
pueblo como por los suyos.  

Y nadie se 
atribuye este honor, sino el que es llamado por 
Dios, como Aarón.  

De igual modo, Cristo no se 
apropió la gloria de ser Sumo Sacerdote, sino 
que se la otorgó el que le dijo:  

<<Tú eres mi hijo, 
yo te he engendrado hoy>>.  

Asimismo, en otro 
lugar, dice también: <<Tú eres sacerdote para 
siempre, según el orden de Melquisedec>>.  

Él, 
en los días de su vida en la tierra, ofreció con 
gran clamor y lágrimas oraciones y súplicas al 
que podía salvarle de la muerte, y fue 
escuchado por su piedad filial, y, aun siendo 
Hijo, aprendió por los padecimientos la 
obediencia.  

Y, llegado a la perfección, se ha 
hecho causa de salvación eterna para todos los 
que le obedecen, ya que fue proclamado por 
Dios Sumo Sacerdote según el orden de 
Melquisedec. 


\fest{23 de abril}{Primera comunión  de San Josemaría}


\lect{Lectura de la primera carta de apóstol san Pedro 
  }{ 2, 2-5}

 Queridos hermanos: Como niños recién 
nacidos que ansían la leche, apeteced vosotros 
la leche espiritual, no adulterada, para que con 
ella crezcáis hacia la salvación, si es que habéis 
gustado qué bueno es el Señor. 

 Acercándoos a 
él, piedra viva, desechada por los hombres, 
pero escogida y preciosa delante de Dios, 
también vosotros -como piedras vivas- sois 
edificados como edificio espiritual para un 
sacerdocio santo, con el fin de ofrecer 
sacrificios espirituales, agradables a Dios por 
medio de Jesucristo. 


\fest{26 de junio}{San Josemaría }


\lect{Lectura de la carta del apóstol san Pablo a los 
Efesios 
  }{  1, 3-6}

 Bendito sea Dios, Padre de nuestro Señor 
Jesucristo, que nos ha bendecido en Cristo con 
toda bendición espiritual en los cielos. Él nos 
eligió en la persona de Cristo antes de la 
creación del mundo para que fuéramos santos 
y sin mancha en su presencia, por el amor; nos 
predestinó a ser sus hijos adoptivos por 
Jesucristo conforme al beneplácito de su 
voluntad, para alabanza y gloria de su gracia, 
con la cual nos hizo gratos en el Amado. 


\fest{2 de octubre}{Fundación del Opus Dei}

\lect{Lectura del Apocalipsis 
  }{ 1, 12-13.17-19 }

 Yo, Juan, volví para ver quién me hablaba; y al 
volverme, vi siete candelabros de oro, y en 
medio de los candelabros como un Hijo de 
hombre, vestido con una túnica hasta los pies, 
y ceñido el pecho con una banda de oro.  

Al 
verle, caí a sus pies como muerto. 

 Él, entonces, 
puso la mano derecha sobre mí, diciendo: <<!`No 
temas! Yo soy el primero y el último, el que 
vive; estuve muerto pero ahora estoy vivo por 
los siglos de los siglos, y tengo las llaves de la 
muerte y del hades. Escribe, por eso, lo que 
has visto, tanto lo presente como lo que va a 
suceder después>>. 

\lect{Lectura del Apocalipsis 
  }{  5, 11-14}

 Yo, Juan, vi y oí un clamor de muchos ángeles 
que rodeaban el trono, de los seres vivos y de 
los ancianos; su número era de miríadas de 
miríadas y millares de millares, que aclamaban 
con gran voz:  

<<Digno es el Cordero inmolado 
de recibir el poder, la riqueza, la sabiduría, la 
fuerza, el honor, la gloria y la alabanza>>. 

 Y a 
toda criatura que existe en el cielo y en la 
tierra, por debajo de la tierra y en el mar, y a 
todo cuanto existe en ellos, les oí decir:  

<<Al 
que está sentado en el trono y al Cordero, la 
alabanza, el honor, la gloria y el poder por los 
siglos de los siglos>>. 

 Y los cuatro seres vivos 
respondían: <<Amén>>. Y los ancianos se 
postraron y adoraron. 


\fest{6 de octubre}{Canonización de San Josemaría}


\lect{Lectura del Apocalipsis 
  }{  7, 9-12}

 Yo, Juan, vi una gran multitud que nadie 
podía contar, de todas las naciones, tribus, 
pueblos y lenguas, de pie ante el trono y ante 
el Cordero, vestidos con túnicas blancas, y con 
palmas en las manos, que gritaban con fuerte 
voz:  

<<!`La salvación viene de nuestro Dios, que 
se sienta sobre el trono, y del Cordero!>>.  

Y 
todos los ángeles estaban de pie alrededor del 
trono, de los ancianos y de los cuatro seres 
vivos, y cayeron sobre sus rostros ante el trono 
y adoraron a Dios, diciendo:  

<<Amén.  La 
bendición, la gloria, la sabiduría, la acción de 
gracias, el honor, el poder y la fortaleza 
pertenecen a nuestro Dios por los siglos de los 
siglos. Amén>>. 


\fest{28 de noviembre}{Erección del Opus Dei en Prelatura Personal}

\lect{Lectura del santo Evangelio según san Lucas 
  }{  10, 21-24}

 En aquel momento Jesús se llenó de gozo en el 
Espíritu Santo y dijo:  

<<Yo te alabo, Padre, 
Señor del cielo y de la tierra, porque has 
ocultado estas cosas a los sabios y prudentes y 
las has revelado a los pequeños. Sí, Padre, 
porque así te ha parecido bien. Todo me lo ha 
entregado mi Padre, y nadie conoce quién es el 
Hijo sino el Padre, ni quién es el Padre sino el 
Hijo, y aquel a quien el Hijo quiera revelarlo>>. 

 
Y volviéndose hacia los discípulos les dijo 
aparte:  

<<Bienaventurados los ojos que ven lo 
que estáis viendo. Pues os aseguro que 
muchos profetas y reyes quisieron ver lo que 
vosotros estáis viendo y no lo vieron; y oír lo 
que estáis oyendo, y no lo oyeron>>. 

\newpage

\subsection{Aniversarios del Prelado}


\lect{Lectura del libro de Jeremías 
  }{  1, 4-8}

 Recibí esta palabra del Señor: <<Antes de 
plasmarte en el seno materno, te conocí, antes 
de que salieras de las entrañas, te consagré, te 
puse como profeta de las naciones>>. 

Respondí: 
<<!`Ay, Señor Dios mío! Si no sé hablar, que soy 
muy joven>>. 

 El Señor me contestó: <<No digas 
que “soy muy joven”, porque allá donde te 
envíe, irás, y todo cuanto te ordene, lo dirás. 
No les tengas miedo, que Yo estoy contigo 
para librarte>>, oráculo del Señor. 

\lect{Lectura del santo Evangelio según san Juan 
  }{  10, 11-18}

 En aquel tiempo dijo Jesús: 
 
 <<Yo soy el buen 
pastor. El buen pastor da su vida por sus 
ovejas. El asalariado, el que no es pastor y al 
que no le pertenecen las ovejas, ve venir el 
lobo, abandona las ovejas y huye --y el lobo las 
arrebata y las dispersa--, porque es asalariado y 
no le importan las ovejas. 

 Yo soy el buen 
pastor, conozco las mías y las mías me 
conocen. Como el Padre me conoce a mí, así yo 
conozco al Padre, y doy mi vida por las ovejas. 
Tengo otras ovejas que no son de este redil, a 
ésas también es necesario que las traiga, y 
oirán mi voz y formarán un solo rebaño, con 
un solo pastor.  

Por eso me ama el Padre, 
porque doy mi vida para tomarla de nuevo. 
Nadie me la quita, sino que yo la doy 
libremente. Tengo potestad para darla y tengo 
potestad para recuperarla. Éste es el mandato 
que he recibido de mi Padre>>. 


\lect{Lectura del santo Evangelio según san Juan 
  }{  15, 9-15}

 En aquel tiempo dijo Jesús a sus discípulos:  


<<Como el Padre me amó, así os he amado yo. 
Permaneced en mi amor. Si guardáis mis 
mandamientos, permaneceréis en mi amor, 
como yo he guardado los mandamientos de mi 
Padre y permanezco en su amor. 

Os he dicho 
esto para que mi alegría esté en vosotros y 
vuestra alegría sea completa. Éste es mi 
mandamiento: que os améis los unos a los 
otros como yo os he amado. Nadie tiene amor 
más grande que el de dar uno la vida por sus 
amigos. Vosotros sois mis amigos si hacéis lo 
que os mando.  

Ya no os llamo siervos, porque 
el siervo no sabe lo que hace su señor; a 
vosotros, en cambio, os he llamado amigos, 
porque todo lo que oí de mi Padre os lo he 
hecho conocer>>. 

\lect{Lectura de la carta del apóstol S. Pablo a los 
Colosenses 
  }{  3,~12-17}

 Hermanos: Como elegidos de Dios, santos y 
amados, revestíos de entrañas de misericordia, 
de bondad, de humildad, de mansedumbre, de 
paciencia, 
sobrellevaos 
mutuamente 
y perdonaos cuando alguno tenga queja contra 
otro; como el Señor os ha perdonado, hacedlo 
así también vosotros: sobre todo, revestíos con 
la caridad, que es el vínculo de la perfección.  

Y 
que la paz de Cristo se adueñe de vuestros 
corazones: a ella habéis sido llamados en un 
solo cuerpo. Y sed agradecidos.  

Que la palabra 
de Cristo habite en vosotros abundantemente. 
Enseñaos con la verdadera sabiduría, animaos 
unos a otros con salmos, himnos y cánticos 
espirituales, cantando agradecidos en vuestros 
corazones. 

 Y todo cuanto hagáis de palabra o 
de obra, hacedlo todo en nombre del Señor 
Jesús, dando gracias a Dios Padre por medio 
de él.

\newpage
\section{Otros aniversarios}
\subsection{Elección del Romano Pontífice}
\lect{Lectura del santo Evangelio según san Mateo 
  }{  6, 13a-15b-19}

 Cuando llegó Jesús a la región de 
Cesarea de Filipo, comenzó a preguntarles a 
sus discípulos:  

<<?`Quién dicen los hombres que 
es el Hijo del Hombre?>>. Él les dijo: <<Y 
vosotros, ?`quién decís que soy yo?>>.  


Respondió Simón Pedro: <<Tú eres el Cristo, el 
Hijo de Dios vivo>>. 

 Jesús le respondió: 
<<Bienaventurado eres, Simón, hijo de Juan, 
porque no te ha revelado eso ni la carne ni la 
sangre, sino mi Padre que está en los cielos. Y 
yo te digo que tú eres Pedro, y sobre esta 
piedra edificaré mi Iglesia, y las puertas del 
infierno no prevalecerán contra ella. Te daré 
las llaves del Reino de los Cielos; y todo lo que 
ates sobre la tierra quedará atado en los cielos, 
y todo lo que desates sobre la tierra, quedará 
desatado en los cielos>>. 

