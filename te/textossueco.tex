\chapter{SAKRAMENTAL ANDAKT OCH VÄLSIGNELSE}
\markboth{SAKRAMENTAL ANDAKT OCH VÄLSIGNELSE}{}

\section{Skriftläsningar}

\subsection{Tiden under året}

\lect{Ur Evangeliet enligt Matteus}
{6, 25-32}
{Därför säger jag er:  bekymra er inte för mat och dryck att leva av eller för kläder att sätta på kroppen. Är inte livet mer än födan och kroppen mer än kläderna?

Se på himlens fåglar, de sår inte, skördar inte och samlar inte i lador, men er himmelske fader föder dem. Är inte ni värda mycket mer än de?
Vem av er kan med sina bekymmer lägga en enda aln till sin livslängd?
Och varför bekymrar ni er för kläder? Se på ängens liljor, hur de växer. De arbetar inte och spinner inte.

Men jag säger er: inte ens Salomo i all sin prakt var klädd som en av dem.
Om nu Gud ger sådana kläder åt gräset på ängen, som i dag finns till och i morgon stoppas i ugnen, skall han då inte ha kläder åt er, ni trossvaga?

Gör er därför inga bekymmer, fråga inte: Vad skall vi äta? Vad skall vi dricka? Vad skall vi ta på oss?

Allt sådant jagar hedningarna efter. Men er himmelske fader vet att ni behöver allt detta.}


\lect{Ur Evangeliet enligt Lukas}{ 22, 14-20}
{När stunden var inne, lade han sig till bords tillsammans med apostlarna.

Han sade till dem: 
”Hur har jag inte längtat efter att få äta denna påskmåltid med er, innan mitt lidande börjar.

Jag säger er: jag kommer inte att äta den igen förrän den får sin fullkomning i Guds rike.”

Man räckte honom en bägare, och han tackade Gud och sade: 
”Ta detta och dela det mellan er.
Jag säger er: från denna stund skall jag inte dricka av det som vinstocken ger förrän Guds rike har kommit.”

Sedan tog han ett bröd, tackade Gud, bröt det och gav åt dem och sade: 
”Detta är min kropp som blir offrad för er. Gör detta till minne av mig.”

Efter måltiden tog han på samma sätt bägaren och sade: 
”Denna bägare är det nya förbundet genom mitt blod, som blir utgjutet för er.}
 
\lect{Ur Evangeliet enligt Johannes}{ 6, 51-58}
{Jag är det levande brödet, som har kommit ner från himlen. Den som äter av det brödet skall leva i evighet. Brödet jag skall ge är mitt kött, jag ger det för att världen skall leva.”

Judarna började då tvista med varandra om hur han kunde ge dem sitt kött att äta.

Jesus svarade: ”Sannerligen, jag säger er: om ni inte äter Människosonens kött och dricker hans blod, äger ni inte livet.
Den som äter mitt kött och dricker mitt blod har evigt liv, och jag skall låta honom uppstå på den sista dagen.
Ty mitt kött är verklig föda, och mitt blod är verklig dryck.
Den som äter mitt kött och dricker mitt blod förblir i mig och jag i honom.
Liksom den levande Fadern har sänt mig och jag lever genom Fadern, skall också den som äter mig leva genom mig.
Detta är brödet som har kommit ner från himlen, ett annat bröd än det som fäderna åt. De dog, men den som äter detta bröd skall leva i evighet.”}



\lect{Ur Evangeliet enligt Johannes}{ 14, 23-27}
{Jesus svarade: ”Om någon älskar mig, bevarar han mitt ord, och min fader skall älska honom, och vi skall komma till honom och stanna hos honom.
Den som inte älskar mig bevarar inte mina ord. Men ordet som ni har hört kommer inte från mig utan från Fadern som har sänt mig.

Detta har jag sagt er, medan jag är kvar hos er.

Men Hjälparen, den heliga anden som Fadern skall sända i mitt namn, han skall lära er allt och påminna er om allt som jag har sagt er.
Frid lämnar jag kvar åt er, min frid ger jag er. Jag ger er inte det som världen ger. Känn ingen oro och tappa inte modet.}

\lect{Ur  Paulus första brev till Korintierna }{ 10, 16-17}
{Välsignelsens bägare som vi välsignar, ger den oss inte gemenskap med Kristi blod? Brödet som vi bryter, ger det oss inte gemenskap med Kristi kropp?

Eftersom brödet är ett enda, är vi - fast många - en enda kropp, för alla får vi vår del av ett och samma bröd.}

\lect{Ur Paulus första brev till Korintierna}{11, 23-26}
{Jag har själv tagit emot från Herren det som jag har fört vidare till er: Den natten då herren Jesus blev förrådd tog han ett bröd,
tackade Gud, bröt det och sade: ”Detta är min kropp som offras för er. Gör detta till minne av mig.”

Likaså tog han bägaren efter måltiden och sade: ”Denna bägare är det nya förbundet genom mitt blod. Var gång ni dricker av den, gör det till minne av mig.”

Var gång ni äter det brödet och dricker den bägaren förkunnar ni alltså Herrens död, till dess han kommer.}


\lect{Ur Johannes första brev}{5, 4-8}
{Eftersom alla som är födda av Gud besegrar världen, och detta är den seger som har besegrat världen: vår tro.
Vem kan besegra världen utom den som tror att Jesus är Guds son?
 
Han är den som kom genom vatten och blod, Jesus Kristus. Inte bara med vattnet utan med både vattnet och blodet. Och Anden är den som vittnar, ty Anden är sanningen.

Det är tre som vittnar:
Anden, vattnet och blodet, och dessa tre är samstämmiga.}

\lect{Ur Johannes Uppenbarelse}{7, 9-12}
{Sedan såg jag, och se: en stor skara som ingen kunde räkna, av alla folk och stammar och länder och språk. De stod inför tronen och Lammet klädda i vita kläder med palmkvistar i sina händer.

Och de ropade med hög röst: ”Frälsningen finns hos vår Gud, som sitter på tronen, och hos Lammet.”

Alla änglarna stod kring tronen och kring de äldste och de fyra varelserna. Och de föll ner på sina ansikten inför tronen och tillbad Gud
och sade: ”Amen. Lovsången och härligheten, visheten och tacksägelsen, äran och makten och kraften tillhör vår Gud i evigheters evighet, amen.”}



\subsection{Advent}


\lect{Ur profeten Jesajas bok}{45, 6b-8}
{Jag är Herren och eljest ingen, jag som danar ljuset och skapar mörkret, jag som ger lyckan och skapar olyckan. Jag, Herren, är den som gör allt detta.
Dryp, ni himlar därovan, och må skyarna låta rättfärdighet strömma ned. Må jorden öppna 
sig, och må dess frukt bli frälsning. Rättfärdighet må den också låta växa upp. Jag, Herren, skapar detta.}


\lect{Ur profeten Jeremias bok}{23, 5-6}
{Se, dagar skall komma, säger Herren, då jag skall låta en rättfärdig avkomling uppstå åt David. Han skall regera som konung och ha framgång, och han skall skaffa rätt och rättfärdighet på jorden.I hans dagar skall Juda bli frälst och Israel bo i trygghet. Och detta skall vara det namn han skall få: Herren vår rättfärdighet.}
 

\lect{Ur Paulus brev till de kristna i Filippi}{4, 4-7}
{Bröder, gläd er alltid i Herren. Än en gång vill jag säga: gläd er. 
 Låt alla människor se hur fördragsamma ni är. Herren är nära. 
 Gör er inga bekymmer, utan när ni åkallar och ber, tacka då Gud och låt honom få veta alla era önskningar. 
 
 Då skall Guds frid, som är mera värd än allt vi tänker, ge era hjärtan och era tankar skydd i Kristus Jesus.}




\subsection{Jultiden}


\lect{Ur Evangeliet enligt Lukas}{2, 8-12.16}
{I samma trakt låg några herdar ute och vaktade sin hjord om natten. 
 Då stod Herrens ängel framför dem och Herrens härlighet lyste omkring dem, och de greps av stor förfäran. 
 
 Men ängeln sade till dem: ”Var inte rädda. Jag bär bud till er om en stor glädje, en glädje för hela folket. 
 I dag har en frälsare fötts åt er i Davids stad, han är Messias, Herren. 
 Och detta är tecknet för er: ni skall finna ett nyfött barn som är lindat och ligger i en krubba.” 
 
 De skyndade i väg och fann Maria och Josef och det nyfödda barnet som låg i krubban.}

\lect{Ur Paulus brev till de kristna i Galatien}{4, 4-7}
{Bröder, när tiden var inne sände Gud sin son, född av en kvinna och född att stå under lagen, 
 för att han skulle friköpa dem som står under lagen och vi få söners rätt. 
 
 Och eftersom ni är söner, har Gud sänt sin sons ande in i vårt hjärta, och den ropar: ”Abba! Fader!” 
 Alltså är du inte längre slav, utan son. Och är du son, har Gud också gjort dig till arvtagare.}


\lect{Ur Paulus brev till de kristna i Filippi}{2, 6-11}
{Han ägde Guds gestalt men vakade inte över sin jämlikhet med Gud, 
 utan avstod från allt och antog en tjänares gestalt då han blev som en av oss. När han till det yttre hade blivit människa, 
 gjorde han sig ödmjuk och var lydig ända till döden, döden på ett kors. 
 
 Därför har Gud upphöjt honom över allt annat och gett honom det namn som står över alla andra namn, 
 för att alla knän skall böjas för Jesu namn, i himlen, på jorden och under jorden, 
 och alla tungor bekänna att Jesus Kristus är herre, Gud fadern till ära.}


\lect{Ur Paulus brev till Titus}{2, 11-14}{Kära bröder, Guds nåd har blivit synlig som en räddning för alla människor. 
 Den lär oss att säga nej till ett gudlöst liv och denna världens begär och att leva anständigt, rättrådigt och fromt i den tid som nu är, 
 medan vi väntar på att vårt saliga hopp skall infrias och vår store gud och frälsare Kristus Jesus träda fram i sin härlighet. 
 
 Han har offrat sig själv för oss, för att friköpa oss från alla synder och göra oss rena, så att vi blir hans eget folk, uppfyllt av iver att göra vad som är gott.}


\subsection{Fastan}

\lect{Ur femte Moseboken}{8, 2-3.14b-16a}{Vid den tiden talade Mose till folket:
Och du skall komma ihåg allt som har skett på den väg Herren, din Gud, nu i fyrtio år har låtit dig vandra i öknen, för att tukta dig och pröva dig, så att han kunde få kännedom om vad som var i ditt hjärta: om du ville hålla hans bud eller inte. 

Ja, han tuktade dig och lät dig hungra, och han gav dig manna att äta, en mat som du förut inte visste om och som inte heller dina fäder visste om, för att han skulle lära dig förstå att människan lever inte endast av bröd utan att hon lever av allt det som utgår av Herrens mun. 

Då må ditt hjärta inte bli högmodigt, så att du glömmer Herren, din Gud, som har fört dig ut ur Egyptens land, ur träldomshuset, och som har lett dig genom den stora och fruktansvärda öknen, bland giftiga ormar och skorpioner, över förtorkad mark, där inget vatten fanns, och som där lät vatten komma ut ur den hårda klippan åt dig och som gav dig manna att äta i öknen, en mat som dina fäder inte visste om.}

\lect{Ur första Kungaboken}{19, 4-8}{Elia gick ut i öknen en dagsresa. Där satte han sig under en ginstbuske. Och han önskade sig döden och sade: ”Det är nog. Ta nu mitt liv, Herre, ty jag är inte förmer än mina fäder.” 

Därefter lade han sig att sova under en ginstbuske. Men se, då rörde en ängel vid honom och sade till honom: ”Stig upp och ät.” 

När han då såg upp, fick han vid sin huvudgärd se ett bröd, sådant som bakas på glödande stenar, och ett krus med vatten. Och han åt och drack och lade sig åter ned. Men Herrens ängel rörde åter vid honom, för andra gången, och sade: ”Stig upp och ät, ty annars blir vägen för lång för dig.” 

Då steg han upp och åt och drack och gick så, styrkt av maten, i fyrtio dagar och fyrtio nätter ända till Guds berg Horeb.}

\lect{Ur Evangeliet enligt Johannes}{6, 26-29}
{Vid den tiden, svarade Jesus:
 ”Sannerligen, jag säger er: ni söker inte efter mig därför att ni har fått se tecken utan därför att ni åt av bröden och blev mätta. Arbeta inte för den föda som är förgänglig utan för den föda som består och skänker evigt liv och som Människosonen skall ge er. Ty på honom har Fadern, Gud själv, satt sitt sigill.”  De frågade då: ”Vad skall vi göra för att utföra Guds verk?” 
 
 Jesus svarade: ”Detta är Guds verk: att ni tror på honom som han har sänt.”}

\subsubsection*{Ur Evangeliet enligt Johannes \hfill{\color{rubrica}\small  6, 30-35} }
%\lect{Ur Evangeliet enligt Johannes}{6, 30-35}
Vid den tiden, sade folket till Jesus:

”Vilket tecken vill du göra, så att vi kan se det och tro på dig? Vad kan du utföra? Våra fäder åt mannat i öknen, så som det står skrivet: Han gav dem bröd från himlen att äta.” 

Jesus svarade: ”Sannerligen, jag säger er: Mose gav er inte brödet från himlen, men min fader ger er det sanna brödet från himlen.  Guds bröd är det bröd som kommer ner från himlen och ger världen liv.” 

De bad honom då: ”Herre, ge oss alltid det brödet.” 

Jesus svarade: ”Jag är livets bröd. Den som kommer till mig skall aldrig hungra, och den som tror på mig skall aldrig någonsin törsta. 



\subsection{Påsktiden}

\lect{Ur Evangeliet enligt Lukas}{24, 13-16.28-34}
{Samma dag var två lärjungar på väg till en by som ligger en mil från Jerusalem och som heter Emmaus. De talade med varandra om allt det som hade hänt. }

Medan de gick där och samtalade och diskuterade, kom Jesus själv och slog följe med dem. Men deras ögon var förblindade och de kände inte igen honom. 
De var nästan framme vid byn dit de skulle, och han såg ut att vilja gå vidare, men de höll kvar honom och sade: ”Stanna hos oss. Det börjar bli kväll, och dagen är snart slut.” 

Då följde han med in och stannade hos dem. 
När han sedan låg till bords med dem, tog han brödet, läste tackbönen, bröt det och gav åt dem. 
Då öppnades deras ögon, och de kände igen honom, men han försvann ur deras åsyn. Och de sade till varandra: ”Brann inte våra hjärtan när han talade till oss på vägen och utlade skrifterna för oss?”

De bröt genast upp och återvände till Jerusalem, där de fann de elva och alla de andra församlade, och dessa sade: ”Herren har verkligen blivit uppväckt och han har visat sig för Simon.”


\lect{Ur Evangeliet enligt Johannes}{20, 19-21}
{På kvällen samma dag, den första i veckan, satt lärjungarna bakom reglade dörrar av rädsla för judarna. Då kom Jesus och stod mitt ibland dem och sade till dem: ”Frid åt er alla.” 

Sedan visade han dem sina händer och sin sida. Lärjungarna blev glada när de såg Herren. 
Jesus sade till dem igen: ”Frid åt er alla. Som Fadern har sänt mig sänder jag er.”}


\lect{Ur Evangeliet enligt Johannes}{20, 26-29}
{En vecka senare var lärjungarna samlade igen, och Tomas var med. Då kom Jesus, trots att dörrarna var reglade, och stod mitt ibland dem och sade: ”Frid åt er alla.” 

Därefter sade han till Tomas: ”Räck hit ditt finger, här är mina händer; räck ut din hand och stick den i min sida. Tvivla inte, utan tro!” Då svarade Tomas: ”Min Herre och min Gud.” 

Jesus sade till honom: ”Du tror därför att du har sett mig. Saliga de som inte har sett men ändå tror.”}


\lect{Ur Apostlagärningarna}{2, 42-47}
{De deltog troget i apostlarnas undervisning och den inbördes hjälpen, i brödbrytandet och bönerna. Alla människor bävade: många under och tecken gjordes genom apostlarna. 

De troende fortsatte att samlas och hade allting gemensamt. De sålde allt vad de ägde och hade och delade ut åt alla, efter vars och ens behov. De höll samman och möttes varje dag troget i templet, och i hemmen bröt de brödet och höll måltid med varandra i jublande, uppriktig glädje. 

De prisade Gud och var omtyckta av hela folket. Och Herren lät var dag nya människor bli frälsta och förena sig med dem.}


\lect{Ur Apostlagärningarna}{2, 42; 4, 32-33}
{De deltog troget i apostlarnas undervisning och den inbördes hjälpen, i brödbrytandet och bönerna. Alla de många som hade kommit till tro var ett hjärta och en själ, och ingen betraktade något av det han ägde som sitt; de hade allt gemensamt. Med stor kraft frambar apostlarna vittnesbördet om att herren Jesus hade uppstått, och de fick alla riklig del av Guds nåd.}


\vspace{25 mm}
\subsection{Högtider och fester som ingår i den romerska kalendern}

\fest{1 januari}{Guds heliga moder Maria}
\lect{Ur Evangeliet enligt Lukas}{2, 16-20}
{Vid den tiden, skyndade de i väg och fann Maria och Josef och det nyfödda barnet som låg i krubban. 

När de hade sett det, berättade de vad som hade sagts till dem om detta barn. 
Alla som hörde det häpnade över vad herdarna sade. 
Maria tog allt detta till sitt hjärta och begrundade det. 

Och herdarna vände tillbaka och prisade och lovade Gud för vad de hade fått höra och se: allt var så som det hade sagts dem.}

	

\fest{6 januari}{Epifania - Herrens Uppenbarelse}
\lect{Ur Evangeliet enligt Matteus}{2, 1-2.7-11}
{När Jesus hade fötts i Betlehem i Judeen på kung Herodes tid, kom några österländska stjärntydare till Jerusalem 
och frågade: ”Var finns judarnas nyfödde kung? Vi har sett hans stjärna gå upp och kommer för att hylla honom.” 

Då kallade Herodes i hemlighet till sig stjärntydarna och förhörde sig noga om hur länge stjärnan hade varit synlig. 
Sedan skickade han dem till Betlehem. ”Bege er dit och ta noga reda på allt om barnet”, sade han, ”och underrätta mig när ni har hittat honom, så att också jag kan komma dit och hylla honom.” 

Efter att ha lyssnat till kungen gav de sig i väg, och stjärnan som de hade sett gå upp gick före dem, tills den slutligen stannade över den plats där barnet var. 
När de såg stjärnan, fylldes de av stor glädje. 

De gick in i huset, och där fann de barnet och Maria, hans mor, och föll ner och hyllade honom. De öppnade sina kistor och räckte fram gåvor: guld och rökelse och myrra.}


\fest{19 mars}{S:t Josef, Jungfru Marias brudgum}
\lect{Ur första Moseboken}{41, 55-57}
{Vid den tiden, när hela Egyptens land började hungra och folket ropade till Farao efter bröd, sade Farao till alla egyptier: ”Gå till Josef, och gör vad han säger er.” 

När nu alltså hungersnöd var över hela landet, öppnade Josef alla förrådshus och sålde säd åt egyptierna. Men hungersnöden blev allt större i Egyptens land. 
Och från alla länder kom man till Josef i Egypten för att köpa säd, ty hungersnöden blev allt större i alla länder.}


\fest{25 mars}{Herrens Bebådelse - Marie Bebådelse dag}
\lect{Ur Evangeliet enligt Johannes}{1, 1-5.9-14}
{I begynnelsen fanns Ordet, och Ordet fanns hos Gud, och Ordet var Gud. 
Det fanns i begynnelsen hos Gud. 

Allt blev till genom det, och utan det blev ingenting till av allt som finns till. 
I Ordet var liv, och livet var människornas ljus. 
Och ljuset lyser i mörkret, och mörkret har inte övervunnit det.
Det sanna ljuset, som ger alla människor ljus, skulle komma in i världen. 
Han var i världen och världen hade blivit till genom honom, men världen kände honom inte. 

Han kom till det som var hans, och hans egna tog inte emot honom. 
Men åt dem som tog emot honom gav han rätten att bli Guds barn, åt alla som tror på hans namn, 
som har blivit födda inte av blod, inte av kroppens vilja, inte av någon mans vilja, utan av Gud.

Och Ordet blev människa och bodde bland oss, och vi såg hans härlighet, en härlighet som den ende sonen får av sin fader, och han var fylld av nåd och sanning.}




\festnoday{Kristi Uppståndelse}
\lect{Ur Evangeliet enligt Johannes}{20, 19-21}
{På kvällen samma dag, den första i veckan, satt lärjungarna bakom reglade dörrar av rädsla för judarna. Då kom Jesus och stod mitt ibland dem och sade till dem: ”Frid åt er alla.” 

Sedan visade han dem sina händer och sin sida. Lärjungarna blev glada när de såg Herren. 
Jesus sade till dem igen: ”Frid åt er alla. Som Fadern har sänt mig sänder jag er.” }


\festnoday{Kristi Himmelsfärdsdag}
\lect{Ur Evangeliet enligt Johannes}{14, 18-21.27-28}
{Vid den tiden sade Jesus till sina lärjungar: Jag skall inte lämna er ensamma, jag skall komma till er. 
Ännu en kort tid, sedan ser världen mig inte längre, men ni skall se mig, eftersom jag lever och ni kommer att leva. 
Den dagen skall ni förstå att jag är i min fader och ni i mig och jag i er. 
Den som har mina bud och håller dem, han älskar mig, och den som älskar mig skall bli älskad av min fader, och jag skall älska honom och visa mig för honom.” 

Frid lämnar jag kvar åt er, min frid ger jag er. Jag ger er inte det som världen ger. Känn ingen oro och tappa inte modet. 
Ni hörde att jag sade till er: Jag går bort och kommer till er igen. Om ni älskade mig, skulle ni vara glada över att jag går till Fadern, ty Fadern är större än jag. }


\festnoday{Pingstdagen}
\lect{Ur Evangeliet enligt Johannes}{14, 23-27}
{Vid den tiden svarade Jesus: ”Om någon älskar mig, bevarar han mitt ord, och min fader skall älska honom, och vi skall komma till honom och stanna hos honom. 
Den som inte älskar mig bevarar inte mina ord. Men ordet som ni har hört kommer inte från mig utan från Fadern som har sänt mig.
Detta har jag sagt er, medan jag är kvar hos er. 
Men Hjälparen, den heliga anden som Fadern skall sända i mitt namn, han skall lära er allt och påminna er om allt som jag har sagt er. 
Frid lämnar jag kvar åt er, min frid ger jag er. Jag ger er inte det som världen ger. Känn ingen oro och tappa inte modet.''}


\festnoday{Heliga Trefaldighets Dag}
\lect{Ur Evangeliet enligt Lukas}{10, 21-24}
{I samma stund fylldes han med jublande glädje genom den heliga anden och sade: ”Jag prisar dig, Fader, himlens och jordens herre, för att du har dolt detta för de lärda och kloka och uppenbarat det för dem som är som barn. Ja, Fader, så har du bestämt. 
Allt har min Fader anförtrott åt mig. Ingen vet vem Sonen är, utom Fadern, och ingen vet vem Fadern är, utom Sonen och den som Sonen vill uppenbara det för.” 
Sedan vände han sig till lärjungarna och sade enbart till dem: ”Saliga de ögon som ser vad ni ser. 
Jag säger er: många profeter och kungar har velat se vad ni ser, men fick inte se det, och velat höra det ni hör, men fick inte höra det.”}


\festnoday{Kristi Kropps och Blods Högtid}
\lect{Ur Vishetens bok}{16, 20-21.25-26}
{Men åt ditt folk delade du i stället ut änglars mat; från himlen bjöd du dem
ett bröd som var färdigt att ätas utan besvär för dem, som förmådde ge all tänkbar njutning och som kunde anpassa sig till vars och ens smak. Det ämne som
kom från dig gav en föreställning om hur ljuv du är mot dina barn. Det fogade
sig efter den ätandes önskan och förvandlades till vad var och en ville ha. 
Därför gjorde sig skapelsen även den gången till tjänare åt din gåva, som kom
med näring åt alla. Den antog alla tänkbara former alltefter vad de bad om och
önskade, för att dina älskade söner, Herre, skulle förstå att det inte är markens gröda som livnär människan med sina olika arter, utan att det är ditt ord
som uppehåller dem som tror på dig. Och detta som elden inte kunde förstöra
smälte för blotta värmen av en flyktig solstråle, för att visa att man måste
tacka dig innan solen går upp och be till dig innan dagen gryr. Ty den otacksammes hopp skall smälta som vinterns rimfrost och flyta bort som utspillt vatten.}


\festnoday{Kristi Kropps och Blods Högtid (Procession)}
\lect{Ur första Krönikeboken}{15, 3-4.15-16;16, 1-2}
{Och David samlade hela Israel till Jerusalem för att hämta Herrens ark upp till den plats som han hade berett åt den. 
Och David samlade ihop Arons barn och leviterna; 
Och så som Mose hade befallt i enlighet med Herrens ord, bar nu Levis barn Guds ark med stänger, som vilade på deras axlar.
Och David sade till de främsta bland leviterna, att de skulle förordna sina bröder sångarna till tjänstgöring med musikinstrument, psaltare, harpor och cymbaler, som de skulle låta ljuda, medan de sjöng glädjesången. 
Sedan de hade fört Guds ark dit in, ställde de den i tältet som David hade slagit upp åt den och bar sedan fram brännoffer och tackoffer inför Guds ansikte. 
När David hade offrat brännoffret och tackoffret, välsignade han folket i Herrens namn.}


\festnoday{Jesu heliga hjärtas dag}
\lect{Ur Evangeliet enligt Johannes}{19, 31-37 }
{Eftersom det var förberedelsedag och kropparna inte fick hänga kvar på korset under sabbaten - det var en stor sabbat - bad judarna Pilatus att de korsfästas benpipor skulle krossas och kropparna tas bort. 
Soldaterna kom därför och krossade benen på dem som var korsfästa tillsammans med Jesus, först på den ene och sedan på den andre. 
Men när de kom till Jesus och såg att han redan var död, krossade de inte hans ben, utan en av soldaterna stack upp sidan på honom med sin lans, och då kom det ut blod och vatten. 
Den som såg det har vittnat om det för att också ni skall tro; hans vittnesbörd är sant, och han vet att han talar sanning. 
Detta skedde för att skriftordet skulle uppfyllas: Inget ben skall brytas på honom. 
Och på ett annat ställe heter det: De skall se på honom som de har genomborrat.}


\fest{24 juni}{Johannes Döparens födelse}
\lect{Ur Evangeliet enligt Johannes}{1, 35 - 39 }
{Nästa dag stod Johannes där igen med två av sina lärjungar. 
När Jesus kom gående, såg Johannes på honom och sade: ”Där är Guds lamm.” 
De båda lärjungarna hörde vad han sade och följde efter Jesus. 
Jesus vände sig om, och då han såg att de följde honom, frågade han vad de ville. De svarade: ”Rabbi (det betyder mästare), var bor du?” 
Han sade: ”Följ med och se!” De gick med honom och såg var han bodde och stannade hos honom den dagen. Det var sent på eftermiddagen.}


\fest{29 juni}{S:t Petrus och S:t Paulus, Apostlar}
\lect{Ur Evangeliet enligt Matteus}{28, 16 - 20}
{
De elva lärjungarna begav sig till Galileen, till det berg dit Jesus hade befallt dem att gå. 
När de fick se honom där föll de ner och hyllade honom, men några tvivlade. 
Då gick Jesus fram till dem och talade till dem: ”Åt mig har getts all makt i himlen och på jorden. 
Gå därför ut och gör alla folk till lärjungar: döp dem i Faderns och Sonens och den heliga Andens namn och lär dem att hålla alla de bud jag har gett er. Och jag är med er alla dagar till tidens slut.”}




\fest{15 augusti}{Jungfru Marias upptagning till himlen}
\lect{Ur Johannes Uppenbarelse}{21, 1 - 4}
{
Och jag såg en ny himmel och en ny jord. Ty den första himlen och den första jorden var borta, och havet fanns inte mer. 
Och jag såg den heliga staden, det nya Jerusalem, komma ner ur himlen, från Gud, redo som en brud som är smyckad för sin man. 
Och från tronen hörde jag en stark röst som sade: ”Se, Guds tält står bland människorna, och han skall bo ibland dem, och de skall vara hans folk, och Gud själv skall vara hos dem, 
och han skall torka alla tårar från deras ögon. Döden skall inte finnas mer, och ingen sorg och ingen klagan och ingen smärta skall finnas mer. Ty det som en gång var är borta.” }


\fest{14 september}{Det Heliga Korsets upphöjelse}
\lect{Ur Paulus brev till de kristna i Filippi}{2, 5 - 11}
{Låt det sinnelag råda hos er som också fanns hos Kristus Jesus. 
Han ägde Guds gestalt men vakade inte över sin jämlikhet med Gud, 
utan avstod från allt och antog en tjänares gestalt då han blev som en av oss. När han till det yttre hade blivit människa, gjorde han sig ödmjuk och var lydig ända till döden, döden på ett kors. 
Därför har Gud upphöjt honom över allt annat och gett honom det namn som står över alla andra namn, för att alla knän skall böjas för Jesu namn, i himlen, på jorden och under jorden, och alla tungor bekänna att Jesus Kristus är Herre, Gud Fadern till ära.}



\fest{29 september}{S:t Mikael, S:t Gabriel och S:t Rafael, ärkeänglar.}
\lect{Ur Johannes  Uppenbarelse}{8, 2-4}
{
Och jag såg de sju änglarna som står inför Gud, och åt dem gavs sju basuner.
Och en annan ängel kom och ställde sig vid altaret med ett rökelsekar av guld, och åt honom gavs mycket rökelse, som han skulle lägga till alla de heligas böner på guldaltaret inför tronen.
Och röken från rökelsen steg ur ängelns hand upp inför Gud tillsammans med de heligas böner.}


\fest{1 november}{Alla helgons dag}
\lect{Ur Johannes Uppenbarelse}{7, 9-12}
{Sedan såg jag, och se: en stor skara som ingen kunde räkna, av alla folk och stammar och länder och språk. De stod inför tronen och Lammet klädda i vita kläder med palmkvistar i sina händer.
Och de ropade med hög röst: ”Frälsningen finns hos vår Gud, som sitter på tronen, och hos Lammet.”
Alla änglarna stod kring tronen och kring de äldste och de fyra varelserna. Och de föll ner på sina ansikten inför tronen och tillbad Gud
och sade: ”Amen. Lovsången och härligheten, visheten och tacksägelsen, äran och makten och kraften tillhör vår Gud i evigheters evighet, amen.”}


\fest{Den sista söndagen i tiden under året}{Kristus Konungens dag}
\lect{Ur Johannes Uppenbarelse}{5, 11-14}
{Och jag såg, och jag hörde rösten av många änglar som stod runt tronen och varelserna och de äldste; deras antal var myriaders myriader, tusen och åter tusen,
och de sade med hög röst: Lammet som blev slaktat är värdigt att ta emot makten och få rikedom och vishet och styrka och ära och härlighet och lovsång.
Och allt skapat i himlen och på jorden och under jorden och på havet och allt som finns där hörde jag säga: Den som sitter på tronen, honom och Lammet tillhör lovsången och äran och härligheten och väldet i evigheters evighet.
Och de fyra varelserna sade: ”Amen.” Och de äldste föll ner och tillbad.}


\fest{8 december}{Jugfru Marias utkorelse och fullkomliga renhet}
\lect{Ur profeten Sakarjas bok}{2, 14-15}
{
Gläd dig och jubla, dotter Sion!
Se, jag kommer
och tar min boning hos dig,
säger Herren.
Många folk skall den dagen
sluta sig till Herren och bli mitt folk.
Jag tar min boning hos dig, och du skall förstå att Herre Sebaot har sänt mig till dig.}

\lect{Ur Paulus brev till de kristna i Rom}{8, 28-30} 
{
Vi vet att Gud på allt sätt hjälper dem som älskar honom att nå det goda, dem som han har kallat efter sin plan.
Ty dem han i förväg har utvalt har han också bestämt till att formas efter hans Sons bild, så att denne skulle vara den förstfödde bland många bröder.
Dem han i förväg har utsett har han också kallat, och dem han har kallat har han också gjort rättfärdiga, och dem han har gjort rättfärdiga, dem har han också skänkt sin härlighet.}




\fest{25 december}{Juldagen, vår Herre Jesu Kristi födelse}
 \lect{Ur Evangeliet enligt Johannes}{1, 1-5. 9-14 }
{
I begynnelsen fanns Ordet, och Ordet fanns hos Gud, och Ordet var Gud.
Det fanns i begynnelsen hos Gud.
Allt blev till genom det, och utan det blev ingenting till av allt som finns till.
I Ordet var liv, och livet var människornas ljus.
Och ljuset lyser i mörkret, och mörkret har inte övervunnit det.
Det sanna ljuset, som ger alla människor ljus, skulle komma in i världen.
Han var i världen och världen hade blivit till genom honom, men världen kände honom inte.
Han kom till det som var hans, och hans egna tog inte emot honom.
Men åt dem som tog emot honom gav han rätten att bli Guds barn, åt alla som tror på hans namn,
som har blivit födda inte av blod, inte av kroppens vilja, inte av någon mans vilja, utan av Gud.
Och Ordet blev människa och bodde bland oss, och vi såg hans härlighet, en härlighet som den ende Sonen får av sin Fader, och han var fylld av nåd och sanning.}


\fest{27 december}{S:t Johannes, apostel och evangelist}
\lect{Ur Johannes första brev}{4, 7-10}
{
Mina kära, låt oss älska varandra, ty kärleken kommer från Gud, och den som älskar är född av Gud och känner Gud.
Men den som inte älskar känner inte Gud, eftersom Gud är kärlek.
Så uppenbarades Guds kärlek hos oss: han sände sin ende Son till världen för att vi skulle få liv genom honom.
Detta är kärleken: inte att vi har älskat Gud utan att han har älskat oss och sänt sin Son som försoningsoffer för våra synder.}




\fest{Söndagen under Juloktaven\\(eller 30 December om ingen söndagen finns)}{Den heliga Familjen}
\lect{Ur Evangeliet enligt Lukas}{2, 4-7}
{
Och Josef, som genom sin härkomst hörde till Davids hus, begav sig från Nasaret i Galileen upp till Judeen, till Davids stad Betlehem,
för att skattskriva sig tillsammans med Maria, sin trolovade, som väntade sitt barn.
Medan de befann sig där var tiden inne för henne att föda,
och hon födde sin son, den förstfödde. Hon lindade honom och lade honom i en krubba, eftersom det inte fanns plats för dem inne i härbärget.}

\newpage
\begin{samepage}
\subsection{Andra Högtider}

{\color{rubrica} \subsubsection{Årsdagar som är relaterade till Opus Deis historia \\och till den helige Josemaría}}

\fest{9 Januari}{S:t Josemarias födelsedag}
\end{samepage}
\lect{Ur profeten Jeremias bok}{1, 4-10}
{Herrens ord kom till mig:
Innan jag formade dig i moderlivet
utvalde jag dig,
innan du kom ut ur modersskötet
gav jag dig ett heligt uppdrag:
att vara profet för folken.
Men jag svarade: ”Nej, Herre, min Gud, jag duger inte till att tala – jag är för ung!” Då sade Herren till mig:
Säg inte att du är för ung
utan gå dit jag sänder dig
och säg det jag befaller dig!
Låt dem inte skrämma dig,
ty jag är med dig
och jag skall rädda dig,
säger Herren.
Och Herren sträckte ut handen, rörde vid min mun och sade:
Jag lägger mina ord i din mun.
I dag ger jag dig makt
över folk och riken.
Du skall rycka upp och vräka omkull,
förstöra och bryta ner,
bygga upp och plantera.}


\fest{14 februari}{Jungfru Maria, den sköna kärlekens Moder}
\lect{Ur Petrus första brev}{2, 2-5. 9-10}
{Som nyfödda barn skall ni längta efter den rena, andliga mjölken, för att växa genom den och bli räddade.
Ni har ju fått smaka Herrens godhet.
När ni kommer till honom, den levande stenen, ratad av människor men utvald av Gud och ärad av honom,
då blir också ni till levande stenar i ett andligt husbygge. Ni blir ett heligt prästerskap och kan frambära andliga offer som Gud vill ta emot tack vare Jesus Kristus.
Men ni är ett utvalt släkte, kungar och präster, ett heligt folk, Guds eget folk som skall förkunna hans storverk. Han har kallat er från mörkret till sitt underbara ljus.
Ni som förut inte var ett folk är nu Guds folk. Ni som förut inte fann barmhärtighet har nu funnit barmhärtighet.}

\pagebreak

\fest{28 mars}{Årsdagen av den helige Josemarias prästvigning}
\lect{Ur brevet till Hebreerna}{5, 1-10}
{Varje överstepräst utses bland människor, och det är människor han får till uppgift att företräda inför Gud, genom att frambära offergåvor för deras synder.
Han kan ha fördrag med de okunniga och vilsegångna, eftersom han själv är behäftad med svaghet
och därför måste frambära syndoffer lika mycket för sig själv som för folket.
Ingen tar sig denna värdighet; han blir kallad av Gud, liksom Aron.
Så är det också med Kristus. Han tog sig inte värdigheten som överstepräst utan fick den av honom som sade: Du är min son, jag har fött dig i dag,
liksom han på ett annat ställe säger: Du är för evigt präst, en sådan som Melkisedek.
Under sitt liv på jorden uppsände han med höga rop och tårar enträgna böner till den som kunde rädda honom från döden, och han blev bönhörd därför att han böjde sig under Guds vilja.
Fast han var Son lärde han sig lyda genom att lida,
och när han hade fullkomnats blev han för alla som lyder honom den som bringar evig frälsning,
av Gud kallad överstepräst, en sådan som Melkisedek.}


\fest{23 april}{Årsdagen av den helige Josemarias första kommunion}
\lect{Ur Petrus första brev}{2, 2-5}
{Som nyfödda barn skall ni längta efter den rena, andliga mjölken, för att växa genom den och bli räddade.
Ni har ju fått smaka Herrens godhet.
När ni kommer till honom, den levande stenen, ratad av människor men utvald av Gud och ärad av honom,
då blir också ni till levande stenar i ett andligt husbygge. Ni blir ett heligt prästerskap och kan frambära andliga offer som Gud vill ta emot tack vare Jesus Kristus.}


\fest{26 juni}{Den helige Josemarias högtid}
\lect{Ur Paulus brev till de kristna i Efesos}{1, 3-6}
{Välsignad är vår herre Jesu Kristi Gud och Fader. Han har välsignat oss med all den andliga välsignelse som genom Kristus finns i himlen,
liksom han före världens skapelse har utvalt oss i honom till att stå heliga och fläckfria inför sig i kärlek.
Han har förutbestämt oss till att få söners rätt genom Jesus Kristus och förenas med honom - det var hans viljas beslut -
till pris och ära för den nåd som han har skänkt oss med sin älskade Son.}


\fest{2 oktober}{De heliga Skyddsänglarna,\\ Årsdagen av Opus Deis grundande}
\lect{Ur Johannes Uppenbarelse}{5, 11-14}
{
Och jag såg, och jag hörde rösten av många änglar som stod runt tronen och varelserna och de äldste; deras antal var myriaders myriader, tusen och åter tusen,
och de sade med hög röst: Lammet som blev slaktat är värdigt att ta emot makten och få rikedom och vishet och styrka och ära och härlighet och lovsång.
Och allt skapat i himlen och på jorden och under jorden och på havet och allt som finns där hörde jag säga: Den som sitter på tronen, honom och Lammet tillhör lovsången och äran och härligheten och väldet i evigheters evighet.
Och de fyra varelserna sade: ”Amen.” Och de äldste föll ner och tillbad.}


\fest{6 oktober}{Årsdagen av den helige Josemarias helgonförklaring}
\lect{Ur Johannes Uppenbarelse}{7, 9-12}
{
Sedan såg jag, och se: en stor skara som ingen kunde räkna, av alla folk och stammar och länder och språk. De stod inför tronen och Lammet klädda i vita kläder med palmkvistar i sina händer.
Och de ropade med hög röst: ”Frälsningen finns hos vår Gud, som sitter på tronen, och hos Lammet.”
Alla änglarna stod kring tronen och kring de äldste och de fyra varelserna. Och de föll ner på sina ansikten inför tronen och tillbad Gud
och sade: ”Amen. Lovsången och härligheten, visheten och tacksägelsen, äran och makten och kraften tillhör vår Gud i evigheters evighet, amen.”}


\fest{28 november}{Årsdagen av Opus Deis upprättande \\som personalprelatur}
\lect{Ur Evangeliet enligt Lukas}{10, 21-24}
{I samma stund fylldes han med jublande glädje genom den Heliga Anden och sade: ”Jag prisar dig, Fader, himlens och jordens herre, för att du har dolt detta för de lärda och kloka och uppenbarat det för dem som är som barn. Ja, Fader, så har du bestämt.
Allt har min Fader anförtrott åt mig. Ingen vet vem Sonen är, utom Fadern, och ingen vet vem Fadern är, utom Sonen och den som Sonen vill uppenbara det för.”
Sedan vände han sig till lärjungarna och sade enbart till dem: ”Saliga de ögon som ser vad ni ser.
Jag säger er: många profeter och kungar har velat se vad ni ser, men fick inte se det, och velat höra det ni hör, men fick inte höra det.”}



%\subsubsection{Andra årsdagar}
\vspace{13 mm}
\begin{center}
{\Large {\color{rubrica}Andra årsdagar}}
\end{center}
\vspace{-4 mm}


\festnoday{Årsdagen av Påvens val}
\lect{Ur Evangeliet enligt Matteus}{16, 13a. 15b-19}
{
När Jesus kom till området kring Caesarea Filippi, frågade han sina lärjungar:
”Vem säger ni att jag är?”
Simon Petrus svarade:”Du är Messias, den levande Gudens son.”
Då sade Jesus till honom: ”Salig är du, Simon Barjona, ty ingen av kött och blod har uppenbarat detta för dig, utan min Fader i himlen.
Och jag säger dig att du är Petrus, Klippan, och på den klippan skall jag bygga min kyrka, och dödsrikets portar skall aldrig få makt över den.
Jag skall ge dig nycklarna till himmelriket. Allt du binder på jorden skall vara bundet i himlen, och allt du löser på jorden skall vara löst i himlen.”}

\newpage
\festnoday{Prelatens Årsdagar*}
\let\thefootnote\relax\footnote{*Dessa läsningar kan användas på följande årsdagar: prelatens födelse - och namnsdag, årsdagen av prelatens val och utnämning samt av hans biskopsvigning (eller prästvigning om han inte biskopvigts).}

\lect{Ur profeten Jeremias bok}{1, 4-8}
{Herrens ord kom till mig:
Innan jag formade dig i moderlivet
utvalde jag dig,
innan du kom ut ur modersskötet
gav jag dig ett heligt uppdrag:
att vara profet för folken.
Men jag svarade: ”Nej, Herre, min Gud, jag duger inte till att tala – jag är för ung!” Då sade Herren till mig:
Säg inte att du är för ung
utan gå dit jag sänder dig
och säg det jag befaller dig!
Låt dem inte skrämma dig,
ty jag är med dig
och jag skall rädda dig,
säger Herren.}


\lect{Ur Evangeliet enligt Johannes}{10, 11-18}
{Jag är den gode herden. Den gode herden ger sitt liv för fåren.
Den som är lejd och inte är herde och inte äger fåren, han överger fåren och flyr när han ser vargen komma, och vargen river dem och skingrar hjorden.
Han är ju lejd och bryr sig inte om fåren.
Jag är den gode herden, och jag känner mina får, och de känner mig,
liksom Fadern känner mig och jag känner Fadern. Och jag ger mitt liv för fåren.
Jag har också andra får, som inte hör till den här fållan. Också dem måste jag leda, och de skall lyssna till min röst, och det skall bli en hjord och en herde.
Fadern älskar mig därför att jag ger mitt liv för att sedan få det tillbaka.
Ingen har tagit det ifrån mig, jag ger det av fri vilja. Jag har rätt att ge det, och jag har rätt att få det tillbaka. Detta har min Fader bestämt för mig.”}


\lect{Ur Evangeliet enligt Johannes}{15, 9-15}
{Liksom Fadern har älskat mig, så har jag älskat er. Bli kvar i min kärlek.
Om ni håller mina bud, blir ni kvar i min kärlek, så som jag har hållit min Faders bud och är kvar i hans kärlek.
Detta har jag sagt er för att min glädje skall vara i er och er glädje bli fullkomlig.
Mitt bud är detta: att ni skall älska varandra så som jag har älskat er.
Ingen har större kärlek än den som ger sitt liv för sina vänner.
Ni är mina vänner, om ni gör vad jag befaller er.
Jag kallar er inte längre tjänare, ty en tjänare vet inte vad hans herre gör. Jag kallar er vänner, därför att jag har låtit er veta allt vad jag har hört av min Fader.}