\selectlanguage{spanish}
\chapter[BENDICIÓN EUCARÍSTICA]{RITO DE LA EXPOSICIÓN Y\\ BENDICIÓN EUCARÍSTICA}

\markboth{BENDICIÓN}{}

\rub{Congregado el pueblo, el ministro cubierto con el paño de hombros, traslada el Sacramento al altar desde el lugar de la reserva. Se puede cantar {\color{black} Pange lingua, O Salutaris hostia} u otro canto adecuado. Si se emplea la custodia, el ministro inciensa al Sacramento. }

\bigskip

\gregorioscore{mu/pangelinguavatican}

\bigskip

\rub{El ministro lee algún texto de la Sagrada Escritura. Se puede tener una homilía o breve exhortación, tras lo cual se guardará un momento de silencio. Después el ministro puede dirigir el rezo de la Visita.}

\rub{Al acabar la adoración  el pueblo canta {\color{black}Tantum ergo} u otro canto eucarístico. Si la exposición tiene lugar con la custodia, el ministro inciensa el Santísimo.}

\vspace{4 mm}

\rub{Tras el himno el ministro canta:}

\bigskip

\va Panem de c\ae lo præstitísti eis (T. P: allelúia).

\vspace{3 mm}

\gregorioscore{mu/panendecoelo}


\rub {La asamblea responde:}

\bigskip

\ra Omne delectaméntum in se habéntem (T. P: allelúia).

\vspace{4 mm}

\gregorioscore{mu/omne}

\rub {Continúa el ministro:}

{\large Orémus.}

%\smallskip

\ltres{D}{} eus, qui nobis sub sacraménto mirábili passiónis tuæ memóriam
reliquísti, tríbue, qu\'\ae sumus, ita nos Córporis et Sánguinis tui
sacra mystéria venerári, ut redemptiónis tuæ fructum in nobis
iúgiter sentiámus. Qui vivis et regnas in s\'\ae cula sæculórum.

\smallskip

\ra Amen.

\vspace{4 mm}

\gregorioscore{mu/omneoremus}

%\newpage


\rub{Dicha la oración, el ministro, tomando el paño de hombros, hace genuflexión, toma la custodia o copón y hace con él en silencio la señal de la cruz sobre el pueblo.
Acabada la bendición, el sacerdote puede incoar las alabanzas de desagravio:}

\vspace{5 mm}

 {\fontsize{16}{16}\selectfont 

{\color{rubrica}B}endito sea {\color{rubrica}D}ios.

\vspace{3 mm}

{\color{rubrica}B}endito sea su santo {\color{rubrica}N}ombre.

\vspace{3 mm}

{\color{rubrica}B}endito sea {\color{rubrica}J}esucristo,  {\color{rubrica}D}ios y {\color{rubrica}H}ombre verdadero.

\vspace{3 mm}

{\color{rubrica}B}endito sea el nombre de {\color{rubrica}J}esús.

\vspace{3 mm}

{\color{rubrica}B}endito sea su {\color{rubrica}S}acratísimo {\color{rubrica}C}orazón.

\vspace{3 mm}

{\color{rubrica}B}endita sea su {\color{rubrica}P}reciosísima {\color{rubrica}S}angre.

\vspace{3 mm}

{\color{rubrica}B}endito sea {\color{rubrica}J}esús en el {\color{rubrica}S}antísimo {\color{rubrica}S}acramento del {\color{rubrica}A}ltar.

\vspace{3 mm}

{\color{rubrica}B}endito sea el {\color{rubrica}E}spíritu {\color{rubrica}S}anto {\color{rubrica}P}aráclito.

\vspace{3 mm}

{\color{rubrica}B}endita sea la excelsa {\color{rubrica}M}adre de {\color{rubrica}D}ios, {\color{rubrica}M}aría {\color{rubrica}S}antísima.

\vspace{3 mm}

{\color{rubrica}B}endita sea su {\color{rubrica}S}anta e {\color{rubrica}I}nmaculada {\color{rubrica}C}oncepción.

\vspace{3 mm}

{\color{rubrica}B}endita sea su gloriosa {\color{rubrica}A}sunción.

\vspace{3 mm}

{\color{rubrica}B}endito sea el nombre de {\color{rubrica}M}aría {\color{rubrica}V}irgen y {\color{rubrica}M}adre.

\vspace{3 mm}

{\color{rubrica}B}endito sea San {\color{rubrica}J}osé, su castísimo {\color{rubrica}E}sposo.

\vspace{3 mm}

{\color{rubrica}B}endito sea {\color{rubrica}D}ios en sus {\color{rubrica}Á}ngeles y en sus {\color{rubrica}S}antos. 

\vspace{3 mm}

Amen.}

\vspace{5 mm}

\rub{Acabada la bendición, el ministro reserva el Sacramento en el sagrario y hace genuflexión. Mientras, el pueblo, si se juzga oportuno, hace alguna aclamación. Finalmente el ministro se retira a la sacristía.}

