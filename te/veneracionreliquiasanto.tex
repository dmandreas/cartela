\selectlanguage{spanish}
\section[Veneración de una reliquia]{Veneración de la reliquia de un santo o de un beato}


Las reliquias\emph{ ex ossibus} de un santo o de un beato pueden venerarse de tres modos:

A) exposición y bendición con la reliquia;

B) veneración de la reliquia, después de la Exposición y bendición con el Santisimo Sacramento;

C) veneracion a continuación de la Santa Misa.

\subsection[Exposición y bendición con la reliquia ]{Exposición y bendición con la reliquia de un santo o de un beato.}

1. El  celebrante\footnote{Revestido con sobrepelliz o alba, estola y capa pluvial blanca. El relicario estará preparado en la sacristía, sobre la mesa o repisa donde se colocan los vasos sagrados.} 
 toma el relicario\footnote{Lo lleva cubierto con el cubrerrelicario, pero sin velo humeral.} y, precedido por los ayudantes, se dirige hacia el oratorio\footnote{Cuando por las características del oratorio, el número de personas disponibles u otros motivos, se desee dar más solemnidad a la ceremonia, se puede prever que otro sacerdote haga de maestro de ceremonias, y que participen varios hacheros con las vestes académicas. En este caso, el celebrante sale de la sacristía sin llevar la reliquia, precedido por el turiferario y los hacheros. Al llegar al presbiterio, hacen la reverencia acostumbrada al Santísimo o al altar, según los casos. A continuación, el maestro de ceremonias, revestido con sobrepelliz y estola blanca, traslada la reliquia, precedido por dos hacheros o por un ayudante con el farol de dos velas.}.

2. Cuando entra en el oratorio, todos se ponen de pie. El coro y el pueblo cantan un himno apropiado\footnote{Puede ser suficiente cantar solo la antífona \emph{Lauda lerusalem}, un versículo del salmo, y otra vez la antífona.}.

3.	Al llegar al presbiterio, el celebrante deja directamente la reliquia en el 
altar\footnote{Si no hay sagrario en el altar, la reliquia se pone en el centro; si hay sagrario, en cambio, no ha de ponerse \emph{ante ostiolum tabernaculi}, sino ligeramente a un lado.}, descubre el relicario y, tras la reverencia acostumbrada, baja al plano, y se pone de rodillas en la grada\footnote{En este momento, los asistentes se arrodillan también. }.

4.	Si se desea, se rezan las \emph{Preces} de la Obra, o se hace un breve rato de oración en silencio.

5.	Al terminar, el sacerdote se pone de pie e impone incienso. Recibe el incensario y, permaneciendo de pie, hace reverencia e inciensa la reliquia con dos \emph{ductus} de dos \emph{ictus}. Al terminar, hace de nuevo reverencia\footnote{Durante incensación los asistentes permanecen de rodillas.}.

6. El sacerdote sube al altar, toma el relicario ---sin el velo humeral y sin hacer genuflexión--- y bendice a los tieles con la reliquia, haciendo la señal de la cruz en 
silencio\footnote{Durante la bendicion, no se toca la campanilla ni se inciensa la reliquia.}.

7. Deja el relicario sobre el altar, y allí mismo canta o reza: \emph{Per merita et intercessionem sancti \parent{beati} {\color {rubrica}N.} concedat vobis Dominus gaudium et pacem}. 
Los asistentes responden: \emph{Amen}, y se ponen de pie.

8. A continuacion, el celebrante besa la reliquia. Después la toma y se coloca en lugar adecuado para darla a besar a los asistentes a la ceremonia\footnote{Los sacerdotes y ministros  se acercan los primeros a venerar la reliquia. Si las características del oratorio lo permiten, los sacerdotes lo hacen en el presbiterio; los demás, en la nave. El maestro, o un ayudante, limpia con un purificador de dedos el cristal que protege la reliquia, cada vez que la besan los participantes en la ceremonia.}.

9. Mientras tantos el coro y el pueblo pueden entonar las Letanías de los santos, o algún himno apropiado, por ej., \emph{Christus vincit}, con el salmo correspondiente, etc.

10. Al terminar la veneración, el celebrante deja la reliquia sobre el altar. De pie\footnote{Tambien los ministros y demás asistentes están de pie.}, reza o canta la oración Colecta de la Misa del santo o del beato. Todos responden: \emph{Amen}.

11. Despues, canta: \emph{Divinum auxilium maneat semper nobiscum}, y se contesta: \emph{Amen}.

12. Un ayudante acerca el cubrerrelicario, el celebrante lo coloca y, \emph{per breviorem}, se dirige a la sacristía, como al comienzo, precedido por los ministros\footnote{Si hay maestro de ceremonias, él se encarga del traslado. En este caso, después del \emph{Divinum...} sube al altar, hace reverencia a la reliquia, pone el cubrerrelicario y se dirige a la sacristía, precedido por uno o varios ayudantes (cfr. nota 3). Después, el celebrante y los demás ministros, hacen la genuflexión o reverencia debida al sagrario o al altar \emph{coram populo}, y se van.}.



\subsection[Veneración de la reliquia]{Veneración después de la exposición y bendición con el Santísimo Sacramento}



Tras la reserva del Santísimo, se traslada el relicario al altar\footnote{Según donde esté el relicario (en el retablo del oratorio, en una de las credencias, en la sacristía, etc ), puede trasladarla el mismo celebrante, o bien uno de los ayudantes la acerca al celebrante y este la coloca en el altar.}, y la ceremonia se desarrolla igual que en el caso anterior (A), con las modificaciones que se indican a continuación.

14.	No se rezan las \emph{Preces} de la Obra, para no alargar demasiado la ceremonia.

15.	Después de la incensación, se omite la bendición con la reliquia, pues ya se ha dado antes con el Santísimo Sacramento, y se pasa directamente al rito de la veneración. cfr. A, nn. 7--9.

\subsection[Veneración a continuación de la Santa Misa]{ Veneración a continuación \\* de la Santa Misa}

16.	Si el relicario no esta colocado habitualmente en el retablo o cerca del retablo, puede ser oportuno que, ya durante la Misa, se dejae sobre una de las credencias, o en una mesita cubierta por un paño blanco\footnote{Se entiende que, en este caso, no se colocarán otros objetos en la credencia o en la mesa.}.

17.	Al acabar la Santa Misa, después de besar el altar, el celebrante recibe el relicario ---descubierto--- de manos del ayudante. Lo deja sobre el altar, y dice: \emph{Per merita et intercessionem sancti \parent{beati} {\color {rubrica}N.}...} A continuación, besa la reliquia y la da a besar a los asistentes.

18.	Al terminar, el sacerdote, en el centro del altar, reza la Colecta de la Misa del santo o del beato. Después, el ayudante devuelve el relicario al retablo o a la credencia.

19.	A continuación, el sacerdote y el ayudante hacen genuflexión al Santísimo ---o reverencia al altar, según los casos--- y se retiran a la sacristia.